% changes example 3
\documentclass[ngerman]{article}

\usepackage{babel}
\usepackage[utf8]{inputenc}
\RequirePackage[T1]{fontenc}

% draft = Ausgabe der Änderungen
\usepackage[draft]{changes}

\definechangesauthor[Ekkart Kleinod]{EK}{orange}
\definechangesauthor{Test}{green}

\setlocextension{changes}
\setauthormarkup[left]{(#1)~--~}
\setremarkmarkup{(#2:#1)}

\begin{document}

	Dieses Beispiel zeigt die erweiterten Funktionen.
	Es setzt die Erweiterung der Hilfsdatei auf \texttt{changes}.
	Ein Autorname wird angegeben.
	Die Autorenmarkierung wird links gesetzt.
	Sie besteht aus dem eingeklammerten Autorennamen, der durch Leerzeichen und einen Gedankenstrich von der Änderung abgesetzt ist.
	Die Anmerkung wird in Klammern gesetzt, der Autorname dahinter.

	\listofchanges

	Dieser Text ist nicht modifiziert.

	Hier \added{füge} ich Text anonym \added{ein}.
	Hier \deleted{lösche} ich anonym Text.
	Und an dieser Stelle \replaced{ändere}{alt} ich anonym Text.
	Anonyme \deleted[][Ja!]{Löschung} mit Anmerkung.

	Hier \added[EK]{füge} ich Text als Autor "`EK"' \added[EK]{ein}.
	Hier füge ich \added[EK][Weil ich es kann.]{Text} als Autor "`EK"' mit Begründung ein.

	Hier \deleted[Test][Weil ich es will.]{lösche} ich Text als Autor "`Text"'.

	Test von Zeilenumbrüchen.

	\added[EK]{eingefügt eingefügt eingefügt eingefügt eingefügt eingefügt eingefügt eingefügt eingefügt eingefügt eingefügt eingefügt eingefügt eingefügt eingefügt eingefügt.}

	\deleted[EK]{gelöscht gelöscht gelöscht gelöscht gelöscht gelöscht gelöscht gelöscht gelöscht gelöscht gelöscht gelöscht gelöscht gelöscht gelöscht gelöscht gelöscht gelöscht gelöscht.}

	\replaced[EK]{eingefügt eingefügt eingefügt eingefügt eingefügt eingefügt eingefügt eingefügt eingefügt eingefügt eingefügt eingefügt eingefügt eingefügt eingefügt eingefügt.}
		{gelöscht gelöscht gelöscht gelöscht gelöscht gelöscht gelöscht gelöscht gelöscht gelöscht gelöscht gelöscht gelöscht gelöscht gelöscht gelöscht gelöscht gelöscht gelöscht.}

\end{document}

