%%
%% This is file `changes.example3.tex',
%% generated with the docstrip utility.
%%
%% The original source files were:
%%
%% changes.dtx  (with options: `example:3')
%% 
%% changes.dtx
%% Copyright 2007-2010 Ekkart Kleinod (ekkart@ekkart.de)
%% 
%% This work may be distributed and/or modified under the
%% conditions of the LaTeX Project Public License, either version 1.3
%% of this license or any later version.
%% The latest version of this license is in
%%  http://www.latex-project.org/lppl.txt
%% and version 1.3 or later is part of all distributions of LaTeX
%% version 2005/12/01 or later.
%% 
%% This work has the LPPL maintenance status `maintained'.
%% The current maintainer of this work is Ekkart Kleinod.
%% 
%% This work consists of the files
%%  changes.drv
%%  changes.dtx
%%  changes.ins
%%  README
%% and the derived files
%%  changes.sty
%%  changes.pdf
%%  changes.example1.tex
%%  changes.example2.tex
%%  changes.example3.tex
%% 
\documentclass[ngerman]{article}

\usepackage{babel}
\usepackage[utf8]{inputenc}
\RequirePackage[T1]{fontenc}

 % draft = Ausgabe der Änderungen
\usepackage[draft]{changes}

\definechangesauthor[Ekkart Kleinod]{EK}{orange}
\definechangesauthor{Test}{green}

\setlocextension{changes}
\setauthormarkup[left]{(#1)~--~}
\setremarkmarkup{(#2:#1)}

\begin{document}

 Dieses Beispiel zeigt die erweiterten Funktionen.
 Es setzt die Erweiterung der Hilfsdatei auf \texttt{changes}.
 Ein Autorname wird angegeben.
 Die Autorenmarkierung wird links gesetzt.
 Sie besteht aus dem eingeklammerten Autorennamen,
 der durch Leerzeichen und einen Gedankenstrich
 von der Änderung abgesetzt ist.
 Die Anmerkung wird in Klammern gesetzt, der Autorname dahinter.

 \listofchanges

 Dieser Text ist nicht modifiziert.

 Hier \added{füge} ich Text anonym \added{ein}.
 Hier \deleted{lösche} ich anonym Text.
 Und an dieser Stelle \replaced{ändere}{alt} ich anonym Text.
 Anonyme \deleted[][Ja!]{Löschung} mit Anmerkung.

 Hier \added[EK]{füge} ich Text als Autor "`EK"' \added[EK]{ein}.
 Hier füge ich \added[EK][Weil ich es kann.]{Text}
 als Autor "`EK"' mit Begründung ein.

 Hier \deleted[Test][Weil ich es will.]{lösche} ich Text
 als Autor "`Text"'.

 Test von Zeilenumbrüchen.
 \added{eingefügt eingefügt eingefügt eingefügt eingefügt
 eingefügt eingefügt eingefügt eingefügt eingefügt eingefügt
 eingefügt eingefügt eingefügt eingefügt eingefügt.}
 \deleted{gelöscht gelöscht gelöscht gelöscht gelöscht gelöscht
 gelöscht gelöscht gelöscht gelöscht gelöscht gelöscht gelöscht
 gelöscht gelöscht gelöscht gelöscht gelöscht gelöscht.}
 \replaced{eingefügt eingefügt eingefügt eingefügt eingefügt
 eingefügt eingefügt eingefügt eingefügt eingefügt eingefügt
 eingefügt eingefügt eingefügt eingefügt eingefügt.}
 {gelöscht gelöscht gelöscht gelöscht gelöscht gelöscht
 gelöscht gelöscht gelöscht gelöscht gelöscht gelöscht gelöscht
 gelöscht gelöscht gelöscht gelöscht gelöscht gelöscht.}

\end{document}
%% Copyright 2007-2010 Ekkart Kleinod
%%
%% End of file `changes.example3.tex'.
