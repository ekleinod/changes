%^^A ---- introduction
\section{Einleitung}

Dieses Paket dient dazu, manuelle Änderungsmarkierung anzubieten.

Verbesserungsvorschläge, Gedanken oder Kritik sind willkommen.
Das Paket wird auf sourceforge gehalten, bitte gehen Sie zu

\url{http://changes.sourceforge.net/}

für Quellcodezugang, Fehler- und Featuretracker, Forum etc.
Wenn Sie mich direkt kontaktieren wollen, mailen Sie bitte an \href{mailto:ekleinod@edgesoft.de}{ekleinod@edgesoft.de}, bitte starten Sie Ihr Mail-Subject mit \texttt{[changes]}.

\begin{quote}
	\small\textsc{README:}
	Das changes-Paket dient zur manuellen Markierung von geändertem Text, insbesondere Einfügungen, Löschungen und Ersetzungen.
	Der geänderte Text wird farbig markiert und, bei gelöschtem Text, durchgestrichen.
	Das Paket ermöglicht die freie Definition von Autoren und deren zugeordneten Farben.
	Es erlaubt zusätzlich die Definition des Autor- und Anmerkungsmarkups.
\end{quote}

%^^A ---- usage
\section{Benutzung des \chpackage{changes}-Pakets}
\label{sec:usage}

In diesem Kapitel wird die typische Nutzung des \chpackage{changes}-Pakets beschrieben.
Dabei wird ein typischer Anwendungsfall geschildert.
Die ausführliche Beschreibung der Paketoptionen und neuen Befehle finden Sie nicht hier, sondern in \autoref{sec:user}.

%^^A ---- user interface
\section{Die Benutzerschnittstelle des \chpackage{changes}-Pakets}
\label{sec:user}

In diesem Kapitel wird die Nutzerschnittstelle des \chpackage{changes}-Pakets vorgestellt, d.\,h.\ alle Optionen und Kommandos.
Jede Option bzw. jedes neue Kommando werden beschrieben.
Wenn Sie die Optionen und Kommandos im Beispiel sehen wollen, sehen Sie bitte in das Beispielverzeichnis unter

\texttt{<texpfad>/doc/latex/changes/examples/}

Die Beispieldateien sind mit der benutzten Option bzw. dem benutzten Kommando benannt.

Um die Änderungsverfolgung zu aktivieren, ist das \chpackage{changes}-Paket wie folgt einzubinden:

\chcommand{usepackage\{changes\}}

bzw.

\chcommand{usepackage[<optionen>]\{changes\}}

Nach der Aktivierung der Änderungsverfolgung schreiben Sie Ihren Text und fügen die Befehle für das Änderungsmanagement ein.
Danach übersetzen Sie Ihr Dokument wie gehabt mit \emph{pdflatex} oder dem \LaTeX-Interpreter Ihrer Wahl.
Wenn Sie eine aktuelle Liste der Änderungen ausgeben wollen, müssen Sie das Dokument zweimal übersetzen: im ersten Lauf wird die Anzahl der Änderungen gesammelt, im zweiten Lauf wird damit die Liste der Änderungen aktualisiert.

%^^A -- options
\subsection{Paketoptionen}
\label{sec:user:options}

\subsubsection{draft}

Die \choption{draft}-Option bewirkt, dass alle Änderungen markiert werden.
Die Änderungsliste kann durch \chcommand{listofchanges} ausgegeben werden.
Ohne Optionsangabe wird \choption{draft} automatisch eingestellt.

Die Angabe von \choption{draft} in \chcommand{documentclass} wird vom \chpackage{changes}-Paket mitgenutzt.
Die lokale Angabe von \choption{final} überstimmt die Angabe von \choption{draft} in \chcommand{documentclass}.

\chcommand{usepackage[draft]\{changes\}}

\subsubsection{final}

Die \choption{final}-Option bewirkt, dass alle Änderungsmarkierungen ausgeblendet werden und nur noch der korrekte Text ausgegeben wird.
Die Änderungsliste wird ebenfalls unterdrückt.

Die Angabe von \choption{final} in \chcommand{documentclass} wird vom \chpackage{changes}-Paket mitgenutzt.
Die lokale Angabe von \choption{draft} überstimmt die Angabe von \choption{final} in \chcommand{documentclass}.

\chcommand{usepackage[final]\{changes\}}

\subsubsection{markup}

Die \choption{markup}-Option wählt ein vordefiniertes visuelles Markup für geänderten Text.
Das default-Markup wird gewählt, wenn die Option nicht gesetzt wird.
Das mit \choption{markup} gewählte Markup kann mit den spezielleren Optionen \choption{addedmarkup} und/oder \choption{deletedmarkup} geändert werden.

Die folgenden Werte sind erlaubt:
\begin{description}
	\item [\choption{default}] farbige Markierung von zugefügtem Text, gelöschter Text wird durchgestrichen (default-Markup)
	\item [\choption{underlined}] zugefügter Text wird unterstrichen, gelöschter Text wird durchgestrichen
	\item [\choption{bfit}] fetter zugefügter Text, schräger gelöschter Text
	\item [\choption{nocolor}] es werden keine Farben verwendet, zugefügter Text wird unterstrichen, gelöschter Text wird durchgestrichen
\end{description}

Beispiele:

\chcommand{usepackage[markup=default]\{changes\}}\\
\chcommand{usepackage[markup=underlined]\{changes\}}\\
\chcommand{usepackage[markup=bfit]\{changes\}}\\
\chcommand{usepackage[markup=nocolor]\{changes\}}

\subsubsection{addedmarkup, deletedmarkup}

Die \choption{addedmarkup}-Option wählt ein visuelles Markup für zugefügten Text.
Die \choption{deletedmarkup}-Option wählt analog ein visuelles Markup für gelöschten Text.
Das default-Markup wird gewählt, wenn die Option nicht gesetzt wird.
Die Optionen \choption{addedmarkup} und \choption{deletedmarkup} überschreiben das mit \choption{markup} gewählte Markup.

Die folgenden Werte sind erlaubt:
\begin{description}
	\item [\choption{none}] kein Markup -- Beispiel (default-Markup für zugefügten Text)
	\item [\choption{uline}] unterstrichener Text -- \uline{Beispiel}
	\item [\choption{uuline}] doppelt unterstrichener Text -- \uuline{Beispiel}
	\item [\choption{uwave}] gewellt unterstrichener Text -- \uwave{Beispiel}
	\item [\choption{dashuline}] gestrichelt unterstrichener Text -- \dashuline{Beispiel}
	\item [\choption{dotuline}] gepunktet unterstrichener Text -- \dotuline{Beispiel}
	\item [\choption{sout}] durchgestrichener Text -- \sout{Beispiel} (default-Markup für gelöschten Text)
	\item [\choption{xout}] schräg durchgestrichener Text -- \xout{Beispiel}
	\item [\choption{bf}] fetter Text -- \textbf{Beispiel}
	\item [\choption{it}] italic Text -- \textit{Beispiel}
	\item [\choption{sl}] schräger Text -- \textsl{Beispiel}
	\item [\choption{em}] hervorgehobener Text -- \emph{Beispiel}
\end{description}

Beispiele:

\chcommand{usepackage[addedmarkup=none]\{changes\}}\\
\chcommand{usepackage[addedmarkup=uline]\{changes\}}\\
\chcommand{usepackage[deletedmarkup=sout]\{changes\}}\\
\chcommand{usepackage[deletedmarkup=xout]\{changes\}}\\
\chcommand{usepackage[deletedmarkup=uwave]\{changes\}}

\subsubsection{authormarkup}

Die \choption{authormarkup}-Option wählt ein visuelles Markup für die Autor-Identifizierung.
Das default-Markup wird gewählt, wenn die Option nicht gesetzt wird.

Die folgenden Werte sind erlaubt:
\begin{description}
	\item [\choption{superscript}] hochgestellter Text -- Text\textsuperscript{Beispiel} (default-Markup)
	\item [\choption{subscript}] tiefgestellter Text -- Text\textsubscript{Beispiel}
	\item [\choption{brackets}] Text in Klammern -- Text(Beispiel)
	\item [\choption{footnote}] Text in einer Fußnote -- Text\footnote{Beispiel}
\end{description}

Beispiele:

\chcommand{usepackage[authormarkup=superscript]\{changes\}}\\
\chcommand{usepackage[authormarkup=subscript]\{changes\}}\\
\chcommand{usepackage[authormarkup=brackets]\{changes\}}\\
\chcommand{usepackage[authormarkup=footnote]\{changes\}}

\subsubsection{authormarkupposition}

Die \choption{authormarkupposition}-Option gibt an, wo die Autor-Identifizierung gesetzt wird.
Der default-Wert wird gewählt, wenn die Option nicht gesetzt wird.

Die folgenden Werte sind erlaubt:
\begin{description}
	\item [\choption{right}] rechts vom Text -- Text\textsuperscript{Beispiel} (default value)
	\item [\choption{left}] links vom Text -- \textsuperscript{Beispiel}Text
\end{description}

Beispiele:

\chcommand{usepackage[authormarkupposition=right]\{changes\}}\\
\chcommand{usepackage[authormarkupposition=left]\{changes\}}

\subsubsection{authormarkuptext}

Die \choption{authormarkuptext}-Option gibt an, was für die Autor-Identifizierung genutzt wird.
Der default-Wert wird gewählt, wenn die Option nicht gesetzt wird.

Die folgenden Werte sind erlaubt:
\begin{description}
	\item [\choption{id}] Autoren-ID -- Text\textsuperscript{ID} (default-Wert)
	\item [\choption{name}] Autorenname -- Text\textsuperscript{Autorenname}
\end{description}

Beispiele:

\chcommand{usepackage[authormarkuptext=id]\{changes\}}\\
\chcommand{usepackage[authormarkuptext=name]\{changes\}}

\subsubsection{ulem}

Optionen für das \chpackage{ulem}-Paket können als Parameter der \choption{ulem}-Option angegeben werden.
Zwei oder mehr Optionen müssen in geschweifte Klammern gesetzt werden.

\chcommand{usepackage[ulem=normalem]\{changes\}}\\
\chcommand{usepackage[ulem=\{normalem,normalbf\}]\{changes\}}

\subsubsection{xcolor}

Optionen für das \chpackage{xcolor}-Paket können als Parameter der \choption{xcolor}-Option angegeben werden.
Zwei oder mehr Optionen müssen in geschweifte Klammern gesetzt werden.

\chcommand{usepackage[xcolor=dvipdf]\{changes\}}\\
\chcommand{usepackage[xcolor=\{dvipdf,gray\}]\{changes\}}

%^^A ---- change management

\subsection{Änderungsmanagement}

\subsubsection{\chcommand{added}}
\DescribeMacro{\added}

Der Befehl \chcommand{added} markiert zugefügten Text.
Der neue Text wird als notwendiges Argument in geschweiften Klammern übergeben.
Optional können eine Autoren-ID sowie eine Anmerkung übergeben werden.
Die Autoren-ID muss mit einer mit dem \chcommand{definechangesauthor}-Befehl definierten ID übereinstimmen.
Soll nur eine Anmerkung (ohne Autor) eingegeben werden, so ist statt des Autors ein leeres Argument zu übergeben.
\begin{einspiel}
\>\chcommand{added[\meta{Autor-ID}][\meta{Anmerkung}]\{\meta{neuer Text}\}}
\end{einspiel}
\begin{einspiel}[true]
\>\texttt{Das ist \chcommand{added}[EK]\{neuer\} Text.}\\
\>Das ist \added[EK]{neuer} Text.\\
\>\texttt{Das ist \chcommand{added}[][anonym]\{neuer\} Text.}\\
\>Das ist \added[][anonym]{neuer} Text.
\end{einspiel}

\subsubsection{\chcommand{deleted}}
\DescribeMacro{\deleted}

Der Befehl \chcommand{deleted} markiert gelöschten Text.
Argumente: siehe \chcommand{added}.
\begin{einspiel}
\>\chcommand{deleted[\meta{Autor-ID}][\meta{Anmerkung}]\{\meta{gelöschter Text}\}}
\end{einspiel}
\begin{einspiel}[true]
\>\texttt{Das ist \chcommand{deleted}[][obsolet]\{schlechter\} Text.}\\
\>Das ist \deleted[][obsolet]{schlechter} Text.
\end{einspiel}

\subsubsection{\chcommand{replaced}}
\DescribeMacro{\replaced}

Der Befehl \chcommand{replaced} markiert geänderten Text.
Notwendige Argumente sind der neue sowie der alte Text.
Optionale Argumente: siehe \chcommand{added}.
\begin{einspiel}
\>\chcommand{replaced[\meta{Autor-ID}][\meta{Anmerkung}]\{\meta{neuer Text}\}\{\meta{alter Text}\}}
\end{einspiel}
\begin{einspiel}[true]
\>\texttt{Das ist \chcommand{replaced}[EK]\{schöner\}\{schlechter\} Text.}\\
\>Das ist \replaced[EK]{schöner}{schlechter} Text.
\end{einspiel}

\subsubsection{\chcommand{listofchanges}}
\DescribeMacro{\listofchanges}

Der Befehl \chcommand{listofchanges} gibt eine Liste der Änderungen aus.
Im ersten \LaTeX-Lauf wird eine Hilfsdatei angelegt, deren Daten im zweiten Durchlauf eingebunden werden.
Für eine aktuelle Liste der Änderungen sind daher zwei \LaTeX-Läufe notwendig.
\begin{einspiel}
	\>\chcommand{listofchanges}
\end{einspiel}

%^^A ---- Author management

\subsection{Autorenverwaltung}

\subsubsection{\chcommand{definechangesauthor}}
\DescribeMacro{\definechangesauthor}

Der Befehl \chcommand{definechangesauthor} definiert einen neuen Autor für Änderungen.
Es muss eine eindeutige Autor-ID angegeben werden, die keine Sonder- oder Leerzeichen enthalten darf.
Optional kann eine Farbe und ein Name angegeben werden.
Wird keine Farbe angegeben, wird schwarz genutzt.
Der Autorenname wird in der Änderungsliste benutzt und im Markup, wenn die entsprechende Option gesetzt ist.
\begin{einspiel}
	\>\chcommand{definechangesauthor[name=\{\meta{Autor-Name}\}, color=\{\meta{Farbe}\}]\{\meta{Autor-ID}\}}
\end{einspiel}

\begin{einspiel}[true]
	\>\chcommand{definechangesauthor\{EK\}}\\
	\>\chcommand{definechangesauthor[color=orange]\{EK\}}\\
	\>\chcommand{definechangesauthor[name=\{Ekkart Kleinod\}]\{EK\}}\\
	\>\chcommand{definechangesauthor[name=\{Ekkart Kleinod\}, color=orange]\{EK\}}
\end{einspiel}

%^^A ---- Adaptation of the output
\subsection{Anpassung der Ausgabe}

\subsubsection{\chcommand{setauthormarkup}}
\DescribeMacro{\setauthormarkup}

Der Befehl \chcommand{setauthormarkup} legt fest, wie der Autor im Text angezeigt wird.
Ohne andere Definition gilt, dass der Autor rechts von den Änderungen hochgestellt erscheint.

Werte für Position (optional): \emph{left} == links von den Änderungen; alles andere: rechts\\
Werte für Definition: beliebige \LaTeX-Befehle, der Autorenname wird mit "`\#1"' gekennzeichnet.
\begin{einspiel}
	\>\chcommand{setauthormarkup[\meta{position}]\{\meta{definition}\}}
\end{einspiel}
\begin{einspiel}[true]
	\>\chcommand{setauthormarkup\{(\#1)\}}\\
	\>\chcommand{setauthormarkup[left]\{(\#1)\textasciitilde{}-{}-\textasciitilde{}\}}\\
	\>\chcommand{setauthormarkup\{}\chcommand{marginpar\{\#1\}\}}\\
	\>\chcommand{setauthormarkup[right]\{\}}
\end{einspiel}

\subsubsection{\chcommand{setremarkmarkup}}
\DescribeMacro{\setremarkmarkup}

Der Befehl \chcommand{setremarkmarkup} legt fest, wie die Anmerkungen im Text angezeigt werden.
Ohne andere Definition gilt, dass die Anmerkungen als Fußnote mit farbigem Text gesetzt werden.

Werte für Definition: beliebige \LaTeX-Befehle, die Autor-ID wird mit "`\#1"' benutzt, der Anmerkungstext mit "`\#2"'.
Über die Autor-ID kann mit \texttt{Changes@Color\#1} die Farbe des Autors benutzt werden.
\begin{einspiel}
	\>\chcommand{setremarkmarkup\{\meta{definition}\}}
\end{einspiel}
\begin{einspiel}[true]
	\>\chcommand{setremarkmarkup\{(\#2:\#1)\}}\\
	\>\chcommand{setremarkmarkup\{\chcommand{footnote}\{\#1:\chcommand{textcolor\{Changes@Color\#1\}}\{\#2\}\}\}}
\end{einspiel}

\subsubsection{\chcommand{setlocextension}}
\DescribeMacro{\setlocextension}

Der Befehl \chcommand{setlocextension} legt das Suffix der Hilfsdatei für die Änderungsliste (loc-Datei\footnote{%
	"`loc"' steht dabei für "`list of changes"'.
}) fest.
Ohne andere Definition gilt das Suffix "`\texttt{loc}"'.
Das Beispiel würde für "`\texttt{foo.tex}"' Hilfsdateien erzeugen, die "`\texttt{foo.changes}"' statt des Standardnamens "`\texttt{foo.loc}"' heißen.
\begin{einspiel}
	\>\chcommand{setlocextension\{\meta{extension}\}}
\end{einspiel}
\begin{einspiel}[true]
	\>\chcommand{setlocextension\{changes\}}
\end{einspiel}

%^^A ---- packages
\subsection{Benötigte Pakete}
\label{sec:user:packages}

Das \chpackage{changes}-Paket bindet bereits Pakete ein, die für das Paket notwendig sind.
Eine genauere Beschreibung der einzelnen Pakete ist in der Dokumentation der Pakete selbst zu finden.

Die folgenden Pakete sind zwingend notwendig müssen für die Nutzung des \chpackage{changes}-Pakets installiert sein:
\begin{description}
	\item [ifthen] stellt eine verbesserte \texttt{if}-Abfrage sowie eine \texttt{while}-Schleife zur Verfügung
	\item [xkeyval] Eingabe von Optionen mit Werteübergabe
\end{description}

Die folgenden Pakete sind manchmal notwendig und müssen installiert sein, wenn sie genutzt werden:
\begin{description}
	\item [pdfcolmk] wird geladen, wenn farbiger Text genutzt wird (default Markup); löst das Problem farbigen Texts über Seitenumbrüche hinweg (bei pdflatex)
	\item [ulem] wird geladen, wenn Text durchgestrichen oder ausge-x-t wird (default Markup); Durchstreichen von Texten
	\item [xcolor] wird geladen, wenn farbiger Text genutzt wird (default Markup); farbige Markierung von Texten
\end{description}


%^^A ---- Authors
\section{Autoren}
\label{sec:authors}

Am \chpackage{changes}-Paket haben mehrere Autoren mitgewirkt.
Dies sind in alphabetischer Reihenfolge:
\begin{itemize}
	\item Chiaradonna, Silvano
	\item Giovannini, Daniele
	\item Kleinod, Ekkart
	\item Wölfel, Philipp
	\item Wolter, Steve
\end{itemize}



%^^A ---- Versions
\section{Versionen}
\label{sec:versions}

\minisec{Version 0.6.0}

Datum: ??.\,0?.~2011
\begin{itemize}
	\item Italienische Übersetzungen der captions von Daniele Giovannini
	\item neues Nutzerinterface für das Setzen von Optionen sowie die Definition von Markup und Autoren
\end{itemize}

\minisec{Version 0.5.4}

Datum: 25.\,04.~2011
\begin{itemize}
	\item Auslagerung der Nutzerdokumentation in eigene Datei
	\item Änderung der default-Sprache zu Englisch
	\item neues Script, um die \chpackage{changes}-Befehle zu löschen von Silvano Chiaradonna
\end{itemize}

\minisec{Version 0.5.3}

Datum: 22.\,11.~2010
\begin{itemize}
\item Dokumentoptionen von \chcommand{documentclass} werden ebenfalls genutzt (Vorschlag und Code von Steve Wolter)
\end{itemize}

\minisec{Version 0.5.2}

Datum: 10.\,10.~2007
\begin{itemize}
	\item Paketoptionen der Pakete \chpackage{pdfcolmk}, \chpackage{ulem}, and \chpackage{xcolor} werden weitergeleitet
\end{itemize}

\minisec{Version 0.5.1}

Datum: 27.\,08.~2007
\begin{itemize}
	\item gelöschter Text wieder durchgestrichen, Paket \chpackage{ulem} funktioniert; ausgrauen hat nicht funktioniert
\end{itemize}

\minisec{Version 0.5}

Datum: 26.\,08.~2007
\begin{itemize}
	\item keine Nutzung des \chpackage{arrayjob}-Pakets mehr, dadurch Fehler im Zusammenspiel mit \chpackage{array} behoben
	\item auf UTF-8-encoding umgestellt
	\item keine Nutzung des \chpackage{soul}-Pakets mehr, dadurch Fehler im Zusammenspiel UTF-8-encoding behoben
	\item gelöschter Text durch grauen Hintergrund visualisiert (es gibt bisher kein ordentliches Durchstreichen bei UTF-8-Nutzung)
	\item neues optionales Argument für Autorenname
	\item farbige Liste der Änderungen
	\item loc-Format geändert
	\item englische Doku verbessert
\end{itemize}

\minisec{Version 0.4}

Datum: 24.\,01.~2007
\begin{itemize}
	\item \chpackage{pdfcolmk} eingebunden, um Problem mit farbigem Text bei Seitenumbrüchen zu lösen
	\item \chcommand{setremarkmarkup} um Autor-ID erweitert, um Anmerkung farbig setzen zu können
	\item Anmerkungen werden in der Fußnote farbig gesetzt
	\item erste Version für das CTAN
\end{itemize}

\minisec{Version 0.3}

Datum: 22.\,01.~2007
\begin{itemize}
	\item englische Nutzerdokumentation
	\item Befehl \chcommand{changed} ersetzt durch \chcommand{replaced}
	\item verbesserte \choption{final}-Option: kein zusätzlicher Leerraum
\end{itemize}

\minisec{Version 0.2}

Datum: 17.\,01.~2007
\begin{itemize}
	\item Bezeichnungen auch bei fehlendem \chpackage{babel}-Paket eingeführt
	\item \chcommand{setauthormarkup}, \chcommand{setlocextension}, \chcommand{setremarkmarkup} für Einstellungen
	\item Beispieldateien generiert
	\item LPPL eingefügt
\end{itemize}
Beseitigte Fehler
\begin{itemize}
	\item Fehler mit \chpackage{ifthen}-Paketplazierung behoben
	\item bei Liste war immer "`Eingefügt"' eingestellt, behoben
	\item Autorausgabe war buggy (\chcommand{if}-Abfrage nicht einwandfrei)
\end{itemize}

\minisec{Version 0.1}

Datum: 16.\,01.~2007
\begin{itemize}
	\item initiale Version
	\item Befehle \chcommand{added}, \chcommand{deleted} und \chcommand{changed}
\end{itemize}

\section{Weitergabe, Copyright, Lizenz}

Copyright 2007-2011 Ekkart Kleinod (\href{mailto:ekleinod@edgesoft.de}{ekleinod@edgesoft.de})

Dieses Paket darf unter der "`\LaTeX\ Project Public License"' Version~1.3 oder jeder späteren Version weitergegeben und/oder geändert werden.
Die neueste Version dieser Lizenz steht auf\\
\url{http://www.latex-project.org/lppl.txt}\\
Version~1.3 und spätere Versionen sind Teil aller \LaTeX-Distributionen ab Version~2005/12/01.

Dieses Paket besitzt den Status "`maintained"' (verwaltet).
Der aktuelle Verwalter dieses Pakets ist Ekkart Kleinod.

Dieses Paket besteht aus den Dateien

\begin{tabbing}
	mm\=\kill
	\>\texttt{source/latex/changes/changes.drv}\\
	\>\texttt{source/latex/changes/changes.dtx}\\
	\>\texttt{source/latex/changes/changes.ins}\\
	\>\texttt{source/latex/changes/README}\\
	\>\texttt{source/latex/changes/examples.dtx}\\

	\>\texttt{scripts/changes/delcmdchanges.bash}
\end{tabbing}

und den generierten Dateien

\begin{tabbing}
	mm\=\kill
	\>\texttt{doc/latex/changes/changes.english.full.pdf}\\
	\>\texttt{doc/latex/changes/changes.english.short.pdf}\\
	\>\texttt{doc/latex/changes/changes.ngerman.full.pdf}\\
	\>\texttt{doc/latex/changes/changes.ngerman.short.pdf}\\

	\>\texttt{doc/latex/changes/examples/changes.example.*.tex}\\
	\>\texttt{doc/latex/changes/examples/changes.example.*.pdf}\\

	\>\texttt{tex/latex/changes/changes.sty}
\end{tabbing}

%^^A end of user documentation

