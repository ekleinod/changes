% changes example 1
\documentclass{article}

% draft = Ausgabe der Änderungen
\usepackage[draft]{changes}

\definechangesauthor{EK}{orange}

\begin{document}

	Dieses Beispiel zeigt die Grundfunktionen.

	\listofchanges

	Dieser Text ist nicht modifiziert.

	Hier \added{f\"uge} ich Text anonym \added{ein}.
	Hier \deleted{l\"osche} ich anonym Text.
	Und an dieser Stelle \replaced{\"andere}{alt} ich anonym Text.
	Anonyme \deleted[][Ja!]{L\"oschung} mit Anmerkung.

	Hier \added[EK]{f\"uge} ich Text als Autor "EK" \added[EK]{ein}.
	Hier f\"uge ich \added[EK][Weil ich es kann.]{Text}	als Autor "EK" mit Begr\"undung ein.

\end{document}

