%%
%% This is file `changes.example1.tex',
%% generated with the docstrip utility.
%%
%% The original source files were:
%%
%% changes.dtx  (with options: `example:1')
%% 
%% changes.dtx
%% Copyright 2007-2010 Ekkart Kleinod (ekkart@ekkart.de)
%% 
%% This work may be distributed and/or modified under the
%% conditions of the LaTeX Project Public License, either version 1.3
%% of this license or any later version.
%% The latest version of this license is in
%%  http://www.latex-project.org/lppl.txt
%% and version 1.3 or later is part of all distributions of LaTeX
%% version 2005/12/01 or later.
%% 
%% This work has the LPPL maintenance status `maintained'.
%% The current maintainer of this work is Ekkart Kleinod.
%% 
%% This work consists of the files
%%  changes.drv
%%  changes.dtx
%%  changes.ins
%%  README
%% and the derived files
%%  changes.sty
%%  changes.pdf
%%  changes.example1.tex
%%  changes.example2.tex
%%  changes.example3.tex
%% 
\documentclass{article}

 % draft = Ausgabe der Änderungen
\usepackage[draft]{changes}

\definechangesauthor{EK}{orange}

\begin{document}

 Dieses Beispiel zeigt die rudiment\"aren Funktionen.

 \listofchanges

 Dieser Text ist nicht modifiziert.

 Hier \added{f\"uge} ich Text anonym \added{ein}.
 Hier \deleted{l\"osche} ich anonym Text.
 Und an dieser Stelle \replaced{\"andere}{alt} ich anonym Text.
 Anonyme \deleted[][Ja!]{L\"oschung} mit Anmerkung.

 Hier \added[EK]{f\"uge} ich Text als Autor "EK" \added[EK]{ein}.
 Hier f\"uge ich \added[EK][Weil ich es kann.]{Text}
 als Autor "EK" mit Begr\"undung ein.

\end{document}
%% Copyright 2007-2010 Ekkart Kleinod
%%
%% End of file `changes.example1.tex'.
