% changes example 2
\documentclass[ngerman]{article}

\usepackage{babel}
\usepackage[utf8]{inputenc}
\RequirePackage[T1]{fontenc}

% draft = Ausgabe der Änderungen
\usepackage[draft]{changes}

\definechangesauthor{EK}{orange}

\begin{document}

	Dieses Beispiel zeigt die Grundfunktionen	unter Einbeziehung des babel-Pakets.

	\listofchanges

	Dieser Text ist nicht modifiziert.

	Hier \added{füge} ich Text anonym \added{ein}.
	Hier \deleted{lösche} ich anonym Text.
	Und an dieser Stelle \replaced{ändere}{alt} ich anonym Text.
	Anonyme \deleted[][Ja!]{Löschung} mit Anmerkung.

	Hier \added[EK]{füge} ich Text als Autor "`EK"' \added[EK]{ein}.
	Hier füge ich \added[EK][Weil ich es kann.]{Text}	als Autor "`EK"' mit Begründung ein.

\end{document}

