%^^A ---- introduction
\section{Einleitung}

Dieses Paket dient dazu, manuelle Änderungsmarkierung zu ermöglichen.

Verbesserungsvorschläge, Gedanken oder Kritik sind willkommen.
Das Paket wird auf \emph{sourceforge} gehalten, bitte gehen Sie zu

\url{http://changes.sourceforge.net/}


für Quellcodezugang, Fehler- und Featuretracker, Forum etc.
Wenn Sie mich direkt kontaktieren wollen, mailen Sie bitte an \href{mailto:ekleinod@edgesoft.de}{ekleinod@edgesoft.de}
.
Bitte starten Sie Ihr Mail-Subject mit \texttt{[changes]}.

\begin{quote}
	\small\textsc{README:}
	Das changes-Paket dient zur manuellen Markierung von geändertem Text, insbesondere Einfügungen, Löschungen und Ersetzungen.
	Der geänderte Text wird farbig markiert und, bei gelöschtem Text, durchgestrichen.
	Das Paket ermöglicht die freie Definition von Autoren und deren zugeordneten Farben.
	Es erlaubt zusätzlich die Änderung des Änderungs-, Autor- und Anmerkungsmarkups.
\end{quote}


%^^A ---- usage
\section{Benutzung des \chpackage{changes}-Pakets}
\label{sec:usage}

In diesem Kapitel wird die Nutzung des \chpackage{changes}-Pakets beschrieben.
Dabei wird ein typischer Anwendungsfall geschildert.
Die ausführliche Beschreibung der Paketoptionen und neuen Befehle finden Sie nicht hier, sondern in \autoref{sec:user}.

Ausgangslage ist ein Text, an dem Änderungen vorgenommen werden sollen.
Diese Änderungen sollen markiert werden, und zwar für jeden Autor einzeln.
Eine solche Änderungsmarkierung ist z.\,B.\ von WYSIWYG-Textprogrammen wie \emph{LibreOffice}, \emph{OpenOffice} oder \emph{Word} bekannt.

Zu diesem Zweck wurde das \chpackage{changes}-Paket entwickelt.
Das Paket stellt Befehle zur Verfügung, um verschiedene Autoren zu definieren und Text als zugefügt, gelöscht oder geändert zu markieren.
Um das Paket zu nutzen, müssen Sie folgende Schritte ausführen:
\begin{enumerate}
	\item \chpackage{changes}-Paket einbinden
	\item Autoren definieren
	\item Textänderungen markieren
	\item Dokument mit \LaTeX\ setzen
	\item Liste von Änderungen anzeigen lassen
	\item Markierungen entfernen
\end{enumerate}

\minisec{\chpackage{changes}-Paket einbinden}

Um die Änderungsverfolgung zu aktivieren, ist das \chpackage{changes}-Paket wie folgt einzubinden:

\chcommand{usepackage\{changes\}}

bzw.

\chcommand{usepackage[<optionen>]\{changes\}}

Mit den verfügbaren Optionen bestimmen Sie hauptsächlich das Aussehen der Änderungsmarkierungen.
Sie können das Aussehen der Änderungsmarkierungen auch nach Einbinden des \chpackage{changes}-Pakets verändern.

Für Details lesen Sie bitte \autoref{sec:user:options} und \autoref{sec:user:customizingoutput}.

\minisec{Autoren definieren}

Das \chpackage{changes}-Paket stellt einen vordefinierten anonymen Autor zur Verfügung.
Wenn Sie jedoch die Änderungen per Autor\_in verfolgen wollen, müssen Sie die entsprechenden Autor\_innen definieren.
Dies geht wie folgt:

\chcommand{definechangesauthor[<optionen>]\{ID\}}

Über die ID werden der/die Autor\_in und die zugehörigen Textänderungen eindeutig identifiziert.
Optional können Sie einen Namen angeben und dem/der Autor\_in eine eigene Farbe zuweisen.

Für Details lesen Sie bitte \autoref{sec:user:authormanagement}.

\minisec{Textänderungen markieren}

Jetzt ist alles vorbereitet, um den geänderten Text zu markieren.
Benutzen Sie bitte je nach Änderung die folgenden Befehle:

für neu zugefügten Text:\\
\chcommand{added[id=<ID>, remark=<Anmerkung>]\{Text\}}

für gelöschten Text:\\
\chcommand{deleted[id=<ID>, remark=<Anmerkung>]\{Text\}}

für geänderten Text:\\
\chcommand{replaced[id=<ID>, remark=<Anmerkung>]\{neuer Text\}\{alter Text\}}

Die Angabe von Autoren-ID und einer Anmerkung ist optional.

Für Details lesen Sie bitte \autoref{sec:user:changemanagement}.

\minisec{Dokument mit \LaTeX\ setzen}

Nachdem Sie die Änderungen im \LaTeX-Text markiert haben, können Sie sie im erzeugten Dokument sichtbar machen, indem Sie das Dokument ganz normal übersetzen.
Durch die Übersetzung wird der geänderte Text so markiert, wie Sie das mittels der Optionen bzw.\ speziellen Befehle eingestellt haben.

\minisec{Liste von Änderungen anzeigen lassen}

Sie können sich eine Liste der Änderungen ausgeben lassen.
Dies erfolgt mit dem Kommando:

\chcommand{listofchanges[style=<list|summary>]}

Die Ausgabe ist gedacht als Analogon zur Liste von Tabellen oder Abbildungen.

Die Angabe des Stils ist optional, standardmäßig wird \choption{style=list} gewählt.
Um einen schnellen Überblick über Art und Anzahl der Änderungen abhängig von dem/der Autor\_in zu bekommen, verwenden Sie den Befehl mit der Option \choption{style=summary}.

Bei jedem \LaTeX-Lauf werden die Daten für diese Liste in eine Hilfsdatei geschrieben.
Beim nächsten \LaTeX-Lauf werden dann diese Daten genutzt, um die Änderungsliste anzuzeigen.
Daher sind nach jeder Änderung zwei \LaTeX-Läufe notwendig, um eine aktuelle Änderungsliste anzuzeigen.

\minisec{Markierungen entfernen}

Oft ist es der Fall, dass die Änderungen eines Dokuments angenommen oder abgelehnt werden und nach diesem Prozess die Änderungsmarkierungen entfernt werden sollen.
Sie können die Ausgabe der Änderungsmarkierungen per Option beim Einbinden des \chpackage{changes}-Pakets unterdrücken:

\chcommand{usepackage[final]\{changes\}}


\subsection{Verfügbare Skripte}

Für die Entfernung der Markierungen aus dem Quelltext steht ein Script von Silvano Chiaradonna zur Verfügung.
Das Script liegt im Verzeichnis:

\texttt{<texpath>/scripts/changes/}


Das Script entfernt alle Markierungen.
Sie können die zu entfernenden Markierungen selektieren bzw.\ selektieren, indem Sie den interaktiven Modus einschalten.
Der interaktive Modus wird mit dem Skriptparameter \texttt{-i} eingeschaltet.

%^^A ---- limitations
\section{Einschränkungen und Erweiterungsmöglichkeiten}
\label{sec:limitations}

Das \chpackage{changes}-Paket ist sorgfältig programmiert und getestet worden.
Dennoch kann es vorkommen, dass Fehler im Paket sind, dass die Benutzung problematisch ist oder dass eine Funktion fehlt, die Sie gerne hätten.
In diesem Fall gehen Sie bitte zu

\url{http://changes.sourceforge.net/}


Dort können Sie Fehler melden, im Forum um Hilfe fragen oder Tips einstellen.
Sie können dort den Quellcode ansehen und nach Ihren Wünschen ändern bzw.\ erweitern.
Ich werde mich dann bemühen, Ihre Änderungen einzuarbeiten.
Sie können als Co-Autor am Paket mitarbeiten, wenn Sie bei \emph{sourceforge} angemeldet sind.

Sie können mir auch eine Mail schreiben an \href{mailto:ekleinod@edgesoft.de}{ekleinod@edgesoft.de}, in diesem Fall starten Sie bitte Ihr Mail-Subject mit \texttt{[changes]}.

Die Änderungsmarkierung von Text funktioniert recht gut, es können auch ganze Absätze markiert werden.
Die Markierung von mehreren Absätzen gleichzeitig, von Bildern und Tabellen ist nicht möglich.

Fußnoten (die standardmäßige Auszeichnung von Anmerkungen) werden in bestimmten Umgebungen, z.\,B.\ Tabellen oder der \emph{tabbing}-Umgebung, nicht korrekt gesetzt, dort erscheinen also Anmerkungen nicht.
Das kann gelöst werden, indem eine andere Annotation von Anmerkungen definiert wird.

Das Paket bietet Raum für Erweiterungen, die ich ich jedoch nicht selbst programmieren werde (weil mir Zeit und oft auch die Fähigkeit fehlt).
Ich liste hier einige Möglichkeiten auf, eine komplettere Liste finden Sie auf der \emph{sourceforge}-Seite:
\begin{itemize}
	\item Auswahl der anzunehmenden/abzulehnenden Änderungen mit entsprechendem Löschen des Textes
	\item Markierung von mehreren Absätzen
	\item Markierung von Bildern und Tabellen
	\item automatische Markierung anhand von diff-Informationen (unter Berücksichtigung der Einschränkungen bzgl.\ Absätzen, Bildern, etc.)
	\item Übersetzung der sprachabhängigen Texte und der Nutzerdokumentation in andere Sprachen
\end{itemize}



%^^A ---- user interface
\section{Die Benutzerschnittstelle des \chpackage{changes}-Pakets}
\label{sec:user}

In diesem Kapitel wird die Nutzerschnittstelle des \chpackage{changes}-Pakets vorgestellt, d.\,h.\ alle Optionen und Kommandos.
Jede Option bzw. jedes neue Kommando werden beschrieben.
Wenn Sie die Optionen und Kommandos im Beispiel sehen wollen, sehen Sie bitte in das Beispielverzeichnis unter

\texttt{<texpath>/doc/latex/changes/examples/}


Die Beispieldateien sind mit der benutzten Option bzw. dem benutzten Kommando benannt.

Eine Vorbemerkung zum Setzen von ersetztem Text: ersetzter Text wird immer wie folgt gesetzt: \meta{neuer Text}\meta{alter Text}.
Daher gibt es keine Möglichkeit, die Ausgabe ersetzten Texts direkt zu beeinflussen, sondern nur über die Änderung der Ausgabe neuen bzw.\ gelöschten Texts.

%^^A -- options
\subsection{Paketoptionen}
\label{sec:user:options}

\subsubsection{draft}

Die \choption{draft}-Option bewirkt, dass alle Änderungen markiert werden.
Die Änderungsliste kann durch \chcommand{listofchanges} ausgegeben werden.
Diese Option ist automatisch voreingestellt.

Die Angabe von \choption{draft} in \chcommand{documentclass} wird vom \chpackage{changes}-Paket mitgenutzt.
Die lokale Angabe von \choption{final} überstimmt die Angabe von \choption{draft} in \chcommand{documentclass}.

\chcommand{usepackage[draft]\{changes\}} \Corresponds\ \chcommand{usepackage\{changes\}}


\subsubsection{final}

Die \choption{final}-Option bewirkt, dass alle Änderungsmarkierungen ausgeblendet werden und nur noch der korrekte Text ausgegeben wird.
Die Änderungsliste wird ebenfalls unterdrückt.

Die Angabe von \choption{final} in \chcommand{documentclass} wird vom \chpackage{changes}-Paket mitgenutzt.
Die lokale Angabe von \choption{draft} überstimmt die Angabe von \choption{final} in \chcommand{documentclass}.

\chcommand{usepackage[final]\{changes\}}


\subsubsection{markup}

Die \choption{markup}-Option wählt ein vordefiniertes visuelles Markup für geänderten Text.
Das default-Markup wird gewählt, wenn die Option nicht gesetzt wird.
Das mit \choption{markup} gewählte Markup kann mit den spezielleren Optionen \choption{addedmarkup} und/oder \choption{deletedmarkup} geändert werden.

Die folgenden Werte sind erlaubt:
\begin{description}
	\item [\choption{default}] farbige Markierung von zugefügtem Text, gelöschter Text wird durchgestrichen (default-Markup)
	\item [\choption{underlined}] zugefügter Text wird unterstrichen, gelöschter Text wird durchgestrichen
	\item [\choption{bfit}] fetter zugefügter Text, schräger gelöschter Text
	\item [\choption{nocolor}] es werden keine Farben verwendet, zugefügter Text wird unterstrichen, gelöschter Text wird durchgestrichen
\end{description}

\begin{chusage}
		\>\chcommand{usepackage[markup=\meta{markup}]\{changes\}}\\
	\usageexample
		\>\chcommand{usepackage[markup=default]\{changes\}} \Corresponds\ \chcommand{usepackage\{changes\}}\\
		\>\chcommand{usepackage[markup=underlined]\{changes\}}\\
		\>\chcommand{usepackage[markup=bfit]\{changes\}}\\
		\>\chcommand{usepackage[markup=nocolor]\{changes\}}
\end{chusage}



\subsubsection{addedmarkup, deletedmarkup}

Die \choption{addedmarkup}-Option wählt ein vordefiniertes visuelles Markup für zugefügten Text.
Die \choption{deletedmarkup}-Option wählt analog ein vordefiniertes visuelles Markup für gelöschten Text.
Das default-Markup wird gewählt, wenn die Option nicht gesetzt wird.
Die Optionen \choption{addedmarkup} und \choption{deletedmarkup} überschreiben das mit \choption{markup} gewählte Markup.

Die folgenden Werte sind erlaubt:
\begin{description}
	\item [\choption{none}] kein Markup -- Beispiel (default-Markup für zugefügten Text)
	\item [\choption{uline}] unterstrichener Text -- \uline{Beispiel}
	\item [\choption{uuline}] doppelt unterstrichener Text -- \uuline{Beispiel}
	\item [\choption{uwave}] gewellt unterstrichener Text -- \uwave{Beispiel}
	\item [\choption{dashuline}] gestrichelt unterstrichener Text -- \dashuline{Beispiel}
	\item [\choption{dotuline}] gepunktet unterstrichener Text -- \dotuline{Beispiel}
	\item [\choption{sout}] durchgestrichener Text -- \sout{Beispiel} (default-Markup für gelöschten Text)
	\item [\choption{xout}] schräg durchgestrichener Text -- \xout{Beispiel}
	\item [\choption{bf}] fetter Text -- \textbf{Beispiel}
	\item [\choption{it}] italic Text -- \textit{Beispiel}
	\item [\choption{sl}] schräger Text -- \textsl{Beispiel}
	\item [\choption{em}] hervorgehobener Text -- \emph{Beispiel}
\end{description}

\begin{chusage}
		\>\chcommand{usepackage[addedmarkup=\meta{markup}]\{changes\}}\\
	\usageexample
		\>\chcommand{usepackage[addedmarkup=none]\{changes\}} \Corresponds\ \chcommand{usepackage\{changes\}}\\
		\>\chcommand{usepackage[addedmarkup=uline]\{changes\}}\\
\end{chusage}


\begin{chusage}
		\>\chcommand{usepackage[deletedmarkup=\meta{markup}]\{changes\}}\\
	\usageexample
		\>\chcommand{usepackage[deletedmarkup=sout]\{changes\}} \Corresponds\ \chcommand{usepackage\{changes\}}\\
		\>\chcommand{usepackage[deletedmarkup=xout]\{changes\}}\\
		\>\chcommand{usepackage[deletedmarkup=uwave]\{changes\}}
\end{chusage}




\subsubsection{authormarkup}

Die \choption{authormarkup}-Option wählt ein vordefiniertes visuelles Markup für die Autor-Identifizierung.
Das default-Markup wird gewählt, wenn die Option nicht gesetzt wird.

Die folgenden Werte sind erlaubt:
\begin{description}
	\item [\choption{superscript}] hochgestellter Text -- Text\textsuperscript{Beispiel} (default-Markup)
	\item [\choption{subscript}] tiefgestellter Text -- Text\textsubscript{Beispiel}
	\item [\choption{brackets}] Text in Klammern -- Text(Beispiel)
	\item [\choption{footnote}] Text in einer Fußnote -- Text\footnote{Beispiel}
	\item [\choption{none}] keine Autor-Identifizierung
\end{description}

\begin{chusage}
		\>\chcommand{usepackage[authormarkup=\meta{markup}]\{changes\}}\\
	\usageexample
		\>\chcommand{usepackage[authormarkup=superscript]\{changes\}} \Corresponds\ \chcommand{usepackage\{changes\}}\\
		\>\chcommand{usepackage[authormarkup=subscript]\{changes\}}\\
		\>\chcommand{usepackage[authormarkup=brackets]\{changes\}}\\
		\>\chcommand{usepackage[authormarkup=footnote]\{changes\}}
\end{chusage}




\subsubsection{authormarkupposition}

Die \choption{authormarkupposition}-Option gibt an, wo die Autor-Identifizierung gesetzt wird.
Der default-Wert wird gewählt, wenn die Option nicht gesetzt wird.

Die folgenden Werte sind erlaubt:
\begin{description}
	\item [\choption{right}] rechts vom Text -- Text\textsuperscript{Beispiel} (default value)
	\item [\choption{left}] links vom Text -- \textsuperscript{Beispiel}Text
\end{description}

\begin{chusage}
		\>\chcommand{usepackage[authormarkupposition=\meta{markup}]\{changes\}}\\
	\usageexample
		\>\chcommand{usepackage[authormarkupposition=right]\{changes\}} \Corresponds\ \chcommand{usepackage\{changes\}}\\
		\>\chcommand{usepackage[authormarkupposition=left]\{changes\}}
\end{chusage}




\subsubsection{authormarkuptext}

Die \choption{authormarkuptext}-Option gibt an, was für die Autor-Identifizierung genutzt wird.
Der default-Wert wird gewählt, wenn die Option nicht gesetzt wird.

Die folgenden Werte sind erlaubt:
\begin{description}
	\item [\choption{id}] Autoren-ID -- Text\textsuperscript{ID} (default-Wert)
	\item [\choption{name}] Autorenname -- Text\textsuperscript{Autorenname}
\end{description}

\begin{chusage}
		\>\chcommand{usepackage[authormarkuptext=\meta{markup}]\{changes\}}\\
	\usageexample
		\>\chcommand{usepackage[authormarkuptext=id]\{changes\}} \Corresponds\ \chcommand{usepackage\{changes\}}\\
		\>\chcommand{usepackage[authormarkuptext=name]\{changes\}}
\end{chusage}




\subsubsection{ulem}

Optionen für das \chpackage{ulem}-Paket können als Parameter der \choption{ulem}-Option angegeben werden.
Zwei oder mehr Optionen müssen in geschweifte Klammern gesetzt werden.

\begin{chusage}
		\>\chcommand{usepackage[ulem=\meta{options}]\{changes\}}\\
	\usageexample
		\>\chcommand{usepackage[ulem=normalem]\{changes\}}\\
		\>\chcommand{usepackage[ulem=\{normalem,normalbf\}]\{changes\}}
\end{chusage}




\subsubsection{xcolor}

Optionen für das \chpackage{xcolor}-Paket können als Parameter der \choption{xcolor}-Option angegeben werden.
Zwei oder mehr Optionen müssen in geschweifte Klammern gesetzt werden.

\begin{chusage}
		\>\chcommand{usepackage[xcolor=\meta{options}]\{changes\}}\\
	\usageexample
		\>\chcommand{usepackage[xcolor=dvipdf]\{changes\}}\\
		\>\chcommand{usepackage[xcolor=\{dvipdf,gray\}]\{changes\}}
\end{chusage}




%^^A ---- change management

\subsection{Änderungsmanagement}
\label{sec:user:changemanagement}

\subsubsection{\chcommand{added}}
\DescribeMacro{\added}

Der Befehl \chcommand{added} markiert zugefügten Text.
Der neue Text wird als notwendiges Argument in geschweiften Klammern übergeben.
Das optionale Argument enthält Key-Value-Paare für die Angabe von Autor-ID sowie einer Anmerkung.
Die Autor-ID muss mit einer mit dem \chcommand{definechangesauthor}-Befehl definierten ID übereinstimmen.
Enthält die Anmerkung Sonderzeichen oder Leerzeichen, ist sie in geschweifte Klammern einzuschließen.

\begin{chusage}
		\>\chcommand{added[id=\meta{Autor-ID}, remark=\meta{Anmerkung}]\{\meta{neuer Text}\}}\\
	\usageexample
		\>\texttt{Das ist \chcommand{added}[id=EK]\{neuer\} Text.}\\
		\>Das ist \added[id=EK]{neuer} Text.\\
		\>\texttt{Das ist \chcommand{added}[id=EK, remark=\{muss rein\}]\{neuer\} Text.}\\
		\>Das ist \added[id=EK, remark={muss rein}]{neuer} Text.\\
		\>\texttt{Das ist \chcommand{added}[remark=anonym]\{neuer\} Text.}\\
		\>Das ist \added[remark=anonym]{neuer} Text.
\end{chusage}


\subsubsection{\chcommand{deleted}}
\DescribeMacro{\deleted}

Der Befehl \chcommand{deleted} markiert gelöschten Text.
Argumente: siehe \chcommand{added}.

\begin{chusage}
		\>\chcommand{deleted[id=\meta{Autor-ID}, remark=\meta{Anmerkung}]\{\meta{gelöschter Text}\}}\\
	\usageexample
		\>\texttt{Das ist \chcommand{deleted}[remark=obsolet]\{schlechter\} Text.}\\
		\>Das ist \deleted[remark=obsolet]{schlechter} Text.
\end{chusage}


\subsubsection{\chcommand{replaced}}
\DescribeMacro{\replaced}

Der Befehl \chcommand{replaced} markiert geänderten Text.
Notwendige Argumente sind der neue sowie der alte Text.
Optionale Argumente: siehe \chcommand{added}.

\begin{chusage}
		\>\chcommand{replaced[id=\meta{Autor-ID}, remark=\meta{Anmerkung}]\{\meta{neuer Text}\}\{\meta{alter Text}\}}\\
	\usageexample
		\>\texttt{Das ist \chcommand{replaced}[id=EK]\{schöner\}\{schlechter\} Text.}\\
		\>Das ist \replaced[id=EK]{schöner}{schlechter} Text.
\end{chusage}


\subsubsection{\chcommand{listofchanges}}
\DescribeMacro{\listofchanges}

Der Befehl \chcommand{listofchanges} gibt eine Liste oder Zusammenfassung der Änderungen aus.
Im ersten \LaTeX-Lauf wird eine Hilfsdatei angelegt, deren Daten im zweiten Durchlauf eingebunden werden.
Für eine aktuelle Liste der Änderungen sind daher zwei \LaTeX-Läufe notwendig.

Die\marginpar{neu ab v2.0.0} Angabe des Stils ist optional, standardmäßig wird die Liste der Änderungen ausgegeben.
Wenn Sie eine Zusammenfassung ausgeben wollen, geben Sie das Argument \choption{style=summary} an.

\begin{chusage}
		\>\chcommand{listofchanges[style=<list|summary>]}
\end{chusage}



%^^A ---- Author management

\subsection{Autorenverwaltung}
\label{sec:user:authormanagement}

\subsubsection{\chcommand{definechangesauthor}}
\DescribeMacro{\definechangesauthor}

Der Befehl \chcommand{definechangesauthor} definiert einen neuen Autor/eine neue Autorin für Änderungen.
Es muss eine eindeutige Autor-ID angegeben werden, die keine Sonder- oder Leerzeichen enthalten darf.
Optional kann eine Farbe und ein Name angegeben werden.
Wird keine Farbe angegeben, wird schwarz genutzt.
Der Autor\_innenname wird in der Änderungsliste sowie im Markup benutzt, im Markup jedoch nur, wenn die entsprechende Option gesetzt ist.

\begin{chusage}
		\>\chcommand{definechangesauthor[name=\{\meta{author's name}\}, color=\{\meta{color}\}]\{\meta{author's id}\}}\\
	\usageexample
		\>\chcommand{definechangesauthor\{EK\}}\\
		\>\chcommand{definechangesauthor[color=orange]\{EK\}}\\
		\>\chcommand{definechangesauthor[name=\{Ekkart Kleinod\}]\{EK\}}\\
		\>\chcommand{definechangesauthor[name=\{Ekkart Kleinod\}, color=orange]\{EK\}}
\end{chusage}



%^^A ---- Adaptation of the output
\subsection{Anpassung der Ausgabe}
\label{sec:user:customizingoutput}

\subsubsection{\chcommand{setaddedmarkup}}
\DescribeMacro{\setaddedmarkup}

Der Befehl \chcommand{setaddedmarkup} legt fest, wie neuer Text ausgezeichnet wird.
Ohne andere Definition gilt, dass der Text farbig oder je nach Option \choption{markup} bzw.\ \choption{addedmarkup} erscheint.

Werte für Definition: beliebige \LaTeX-Befehle, der neue Text wird mit "`\#1"' eingesetzt.

\begin{chusage}
		\>\chcommand{setaddedmarkup\{\meta{definition}\}}\\
	\usageexample
		\>\chcommand{setaddedmarkup\{}\chcommand{emph\{\#1\}}\}\\
		\>\chcommand{setaddedmarkup\{+++: \#1\}}
\end{chusage}



\subsubsection{\chcommand{setdeletedmarkup}}
\DescribeMacro{\setdeletedmarkup}

Der Befehl \chcommand{setdeletedmarkup} legt fest, wie gelöschter Text ausgezeichnet wird.
Ohne andere Definition gilt, dass der Text durchgestrichen oder je nach Option \choption{markup} bzw.\ \choption{deletedmarkup} erscheint.

Werte für Definition: beliebige \LaTeX-Befehle, der gelöschte Text wird mit "`\#1"' eingesetzt.

\begin{chusage}
		\>\chcommand{setdeletedmarkup\{\meta{definition}\}}\\
	\usageexample
		\>\chcommand{setdeletedmarkup\{}\chcommand{emph\{\#1\}}\}\\
		\>\chcommand{setdeletedmarkup\{---: \#1\}}
\end{chusage}



\subsubsection{\chcommand{setauthormarkup}}
\DescribeMacro{\setauthormarkup}

Der Befehl \chcommand{setauthormarkup} legt fest, wie der Autortext im Text angezeigt wird.
Ohne andere Definition gilt, dass der Autor hochgestellt erscheint.

Werte für Definition: beliebige \LaTeX-Befehle, der Autortext wird mit "`\#1"' eingesetzt.

\begin{chusage}
		\>\chcommand{setauthormarkup\{\meta{definition}\}}\\
	\usageexample
		\>\chcommand{setauthormarkup\{(\#1)\}}\\
		\>\chcommand{setauthormarkup\{(\#1)\textasciitilde{}-{}-\textasciitilde{}\}}\\
		\>\chcommand{setauthormarkup\{}\chcommand{marginpar\{\#1\}\}}
\end{chusage}



\subsubsection{\chcommand{setauthormarkupposition}}
\DescribeMacro{\setauthormarkupposition}

Der Befehl \chcommand{setauthormarkupposition} legt fest, auf welcher Seite der Autor im Text angezeigt wird.
Ohne andere Definition gilt, dass der Autor rechts von den Änderungen erscheint.

Mögliche Werte: \emph{left} == links von den Änderungen; alles andere: rechts

\begin{chusage}
		\>\chcommand{setauthormarkupposition\{\meta{position}\}}\\
	\usageexample
		\>\chcommand{setauthormarkupposition\{left\}}
\end{chusage}




\subsubsection{\chcommand{setauthormarkuptext}}
\DescribeMacro{\setauthormarkuptext}

Der Befehl \chcommand{setauthormarkuptext} legt fest, welche Information des Autors im Text angezeigt wird.
Ohne andere Definition gilt, dass die Autor-ID genutzt wird.

Mögliche Werte: \emph{name} == Autorenname; alles andere: Autor-ID

\begin{chusage}
		\>\chcommand{setauthormarkuptext\{\meta{text}\}}\\
	\usageexample
		\>\chcommand{setauthormarkuptext\{name\}}
\end{chusage}




\subsubsection{\chcommand{setremarkmarkup}}
\DescribeMacro{\setremarkmarkup}

Der Befehl \chcommand{setremarkmarkup} legt fest, wie die Anmerkungen im Text angezeigt werden.
Ohne andere Definition gilt, dass die Anmerkungen als Fußnote gesetzt werden.

Werte für Definition: beliebige \LaTeX-Befehle, die Autor-ID wird mit "`\#1"' benutzt, der Anmerkungstext mit "`\#2"'.
Über die Autor-ID kann mit \texttt{Changes@Color\#1} die Farbe des Autors benutzt werden.

\begin{chusage}
		\>\chcommand{setremarkmarkup\{\meta{definition}\}}\\
	\usageexample
		\>\chcommand{setremarkmarkup\{(\#2 --- \#1)\}}\\
		\>\chcommand{setremarkmarkup\{\chcommand{footnote}\{\#1:\chcommand{textcolor\{Changes@Color\#1\}}\{\#2\}\}\}}
\end{chusage}




\subsubsection{\chcommand{setsocextension}}
\DescribeMacro{\setsocextension}

Der\marginpar{neu ab v2.0.0} Befehl \chcommand{setsocextension} legt das Suffix der Hilfsdatei für die Änderungszusammenfassung (soc-Datei\footnote{%
	"`soc"' steht dabei für "`summary of changes"'.
}) fest.
Ohne andere Definition gilt das Suffix "`\texttt{soc}"'.
Im unten angegebenen Beispiel würde für "`\texttt{foo.tex}"' eine Hilfsdatei erzeugt werden, die "`\texttt{foo.changes}"' statt des Standardnamens "`\texttt{foo.soc}"' hieße.

\begin{chusage}
		\>\chcommand{setsocextension\{\meta{extension}\}}\\
	\usageexample
		\>\chcommand{setsocextension\{changes\}}
\end{chusage}




%^^A ---- other
\subsection{Sonstige neue Befehle}
\label{sec:user:other}

\subsubsection{\chcommand{textsubscript}}
\DescribeMacro{\textsubscript}

\LaTeX\ stellt den Befehl \chcommand{textsuperscript} zur Verfügung, nicht jedoch dessen Gegenstück \chcommand{textsubscript}.
Falls der Befehl nicht bereits definiert ist, wird er durch das \chpackage{changes}-Paket zur Verfügung gestellt.
Ist er bereits definiert, wird er nicht geändert.
\begin{chusage}
		\>\chcommand{textsubscript\{\meta{Text}\}}\\
	\usageexample
		\>\texttt{Jetzt kommt ein \chcommand{textsubscript\{tiefgestellter\}} Text.}\\
		\>Jetzt kommt ein \textsubscript{tiefgestellter} Text.
\end{chusage}


%^^A ---- packages
\subsection{Benötigte Pakete}
\label{sec:user:packages}

Das \chpackage{changes}-Paket bindet bereits Pakete ein, die für die Funktion des Pakets notwendig sind.
Eine genauere Beschreibung der einzelnen Pakete ist in der Dokumentation der Pakete selbst zu finden.

Die folgenden Pakete sind zwingend notwendig und müssen für die Nutzung des \chpackage{changes}-Pakets installiert sein:
\begin{description}
	\item [xifthen] stellt eine verbesserte \texttt{if}-Abfrage sowie eine \texttt{while}-Schleife zur Verfügung
	\item [xkeyval] Eingabe von Optionen mit Werteübergabe
\end{description}

Die folgenden Pakete sind manchmal notwendig und müssen installiert sein, wenn sie über die entsprechende Option genutzt werden:
\begin{description}
	\item [pdfcolmk] wird geladen, wenn farbiger Text genutzt wird (default Markup); löst das Problem farbigen Texts über Seitenumbrüche hinweg (bei pdflatex)
	\item [ulem] wird geladen, wenn Text durchgestrichen oder ausge-x-t wird (default Markup)
	\item [xcolor] wird geladen, wenn farbiger Text genutzt wird (default Markup)
\end{description}


%^^A ---- Authors
\section{Autoren}
\label{sec:authors}

Am \chpackage{changes}-Paket haben mehrere Autoren mitgewirkt.
Dies sind in alphabetischer Reihenfolge:
\begin{itemize}
	\item Chiaradonna, Silvano
	\item Giovannini, Daniele
	\item Kleinod, Ekkart
	\item Wölfel, Philipp
	\item Wolter, Steve
\end{itemize}




%^^A ---- Versions
\section{Versionen}
\label{sec:versions}

\minisec{Version 2.0.0}

Datum: 30.\,06.~2013
\begin{itemize}
	\item "`richtige"' Liste der Änderungen, alte Zusammenfassung jetzt über den optionalen Parameter \choption{style=summary}
	\item Problem mit einigen Sonderzeichen in der Änderungszusammenfassung gelöst
	\item \chcommand{setlocextension} umbenannt in \chcommand{setsocextension}
	\item neues Autormarkup \choption{none}
	\item Scriptbeschreibung um Parameter \texttt{-i} ergänzt
\end{itemize}

\minisec{Version 1.0.0}

Datum: 25.\,04.~2012
\begin{itemize}
	\item Key-Value-Interface für Änderungsmanagement-Kommandos
	\item Fehler (Crash) in Änderungsliste gefixt, der bei Sonderzeichen auftrat
	\item Leerzeichen vor Autor\_innenname in Änderungsliste
	\item Fehlermeldung bei Verwendung einer ungültigen Autor\_innen-ID
\end{itemize}

\minisec{Version 0.6.0}

Datum: 11.\,01.~2012
\begin{itemize}
	\item Italienische Übersetzungen der captions von Daniele Giovannini
	\item neues Nutzerinterface für das Setzen von Optionen sowie die Definition von Markup und Autoren
	\item Restrukturierung und Codeverbesserung
	\item verbesserte Dokumentation mit typischem Anwendungsfall
	\item Beispieldateien für alle Optionen und Befehle
	\item Anmerkungen sind per Default nicht mehr farbig
\end{itemize}

\minisec{Version 0.5.4}

Datum: 25.\,04.~2011
\begin{itemize}
	\item Auslagerung der Nutzerdokumentation in eigene Datei
	\item Änderung der default-Sprache zu Englisch
	\item neues Script, um die \chpackage{changes}-Befehle zu löschen von Silvano Chiaradonna
\end{itemize}

\minisec{Version 0.5.3}

Datum: 22.\,11.~2010
\begin{itemize}
\item Dokumentoptionen von \chcommand{documentclass} werden ebenfalls genutzt (Vorschlag und Code von Steve Wolter)
\end{itemize}

\minisec{Version 0.5.2}

Datum: 10.\,10.~2007
\begin{itemize}
	\item Paketoptionen der Pakete \chpackage{ulem} und \chpackage{xcolor} werden weitergeleitet
\end{itemize}

\minisec{Version 0.5.1}

Datum: 27.\,08.~2007
\begin{itemize}
	\item gelöschter Text wieder durchgestrichen, Paket \chpackage{ulem} funktioniert; ausgrauen hat nicht funktioniert
\end{itemize}

\minisec{Version 0.5}

Datum: 26.\,08.~2007
\begin{itemize}
	\item keine Nutzung des \chpackage{arrayjob}-Pakets mehr, dadurch Fehler im Zusammenspiel mit \chpackage{array} behoben
	\item auf UTF-8-encoding umgestellt
	\item keine Nutzung des \chpackage{soul}-Pakets mehr, dadurch Fehler im Zusammenspiel UTF-8-encoding behoben
	\item gelöschter Text durch grauen Hintergrund visualisiert (es gibt bisher kein ordentliches Durchstreichen bei UTF-8-Nutzung)
	\item neues optionales Argument für Autorenname
	\item farbige Liste der Änderungen
	\item loc-Format geändert
	\item englische Doku verbessert
\end{itemize}

\minisec{Version 0.4}

Datum: 24.\,01.~2007
\begin{itemize}
	\item \chpackage{pdfcolmk} eingebunden, um Problem mit farbigem Text bei Seitenumbrüchen zu lösen
	\item \chcommand{setremarkmarkup} um Autor-ID erweitert, um Anmerkung farbig setzen zu können
	\item Anmerkungen werden in der Fußnote farbig gesetzt
	\item erste Version für das CTAN
\end{itemize}

\minisec{Version 0.3}

Datum: 22.\,01.~2007
\begin{itemize}
	\item englische Nutzerdokumentation
	\item Befehl \chcommand{changed} ersetzt durch \chcommand{replaced}
	\item verbesserte \choption{final}-Option: kein zusätzlicher Leerraum
\end{itemize}

\minisec{Version 0.2}

Datum: 17.\,01.~2007
\begin{itemize}
	\item Bezeichnungen auch bei fehlendem \chpackage{babel}-Paket eingeführt
	\item \chcommand{setauthormarkup}, \chcommand{setlocextension}, \chcommand{setremarkmarkup} für Einstellungen
	\item Beispieldateien generiert
	\item LPPL eingefügt
\end{itemize}
Beseitigte Fehler
\begin{itemize}
	\item Fehler mit \chpackage{ifthen}-Paketplazierung behoben
	\item bei Liste war immer "`Eingefügt"' eingestellt, behoben
	\item Autorausgabe war buggy (\chcommand{if}-Abfrage nicht einwandfrei)
\end{itemize}

\minisec{Version 0.1}

Datum: 16.\,01.~2007
\begin{itemize}
	\item initiale Version
	\item Befehle \chcommand{added}, \chcommand{deleted} und \chcommand{changed}
\end{itemize}


%^^A ---- copyright, license
\section{Weitergabe, Copyright, Lizenz}

Copyright 2007-2012 Ekkart Kleinod (\href{mailto:ekleinod@edgesoft.de}{ekleinod@edgesoft.de}
)


Dieses Paket darf unter der "`\LaTeX\ Project Public License"' Version~1.3 oder jeder späteren Version weitergegeben und/oder geändert werden.
Die neueste Version dieser Lizenz steht auf \url{http://www.latex-project.org/lppl.txt} Version~1.3 und spätere Versionen sind Teil aller \LaTeX-Distributionen ab Version~2005/12/01.

Dieses Paket besitzt den Status "`maintained"' (verwaltet).
Der aktuelle Verwalter dieses Pakets ist Ekkart Kleinod.

Dieses Paket besteht aus den Dateien

\begin{tabbing}
	mm\=\kill
	\>\texttt{source/latex/changes/changes.drv}\\
	\>\texttt{source/latex/changes/changes.dtx}\\
	\>\texttt{source/latex/changes/changes.ins}\\
	\>\texttt{source/latex/changes/examples.dtx}\\
	\>\texttt{source/latex/changes/README}\\
	\>\texttt{source/latex/changes/userdoc/*.tex}\\

	\>\texttt{scripts/changes/delcmdchanges.bash}
\end{tabbing}



und den generierten Dateien

\begin{tabbing}
	mm\=\kill
	\>\texttt{doc/latex/changes/changes.english.full.pdf}\\
	\>\texttt{doc/latex/changes/changes.english.short.pdf}\\
	\>\texttt{doc/latex/changes/changes.ngerman.full.pdf}\\
	\>\texttt{doc/latex/changes/changes.ngerman.short.pdf}\\

	\>\texttt{doc/latex/changes/examples/changes.example.*.tex}\\
	\>\texttt{doc/latex/changes/examples/changes.example.*.pdf}\\

	\>\texttt{tex/latex/changes/changes.sty}
\end{tabbing}



%^^A end of user documentation

