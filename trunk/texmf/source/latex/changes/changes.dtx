% \CheckSum{818}
%
% \iffalse meta-comment
%
%  Copyright (C) 2007-2011
%  Ekkart Kleinod (ekleinod@edgesoft.de)
% --------------------------------------------------------------------------
%
%  This work may be distributed and/or modified under the
%  conditions of the \LaTeX\ Project Public License, either version~1.3
%  of this license or any later version.
%  The latest version of this license is in\\
%   \url{http://www.latex-project.org/lppl.txt}\\
%  and version~1.3 or later is part of all distributions of \LaTeX\
%  version 2005/12/01 or later.
%
%  This work has the LPPL maintenance status "maintained".
%  The current maintainer of this work is Ekkart Kleinod.
%
%  Some code for providing multilanguage documentation was
%  used from the pst-pdf package by Rolf Niepraschk and Hubert Gaesslein.
% \fi
%
% \CharacterTable
%  {Upper-case    \A\B\C\D\E\F\G\H\I\J\K\L\M\N\O\P\Q\R\S\T\U\V\W\X\Y\Z
%   Lower-case    \a\b\c\d\e\f\g\h\i\j\k\l\m\n\o\p\q\r\s\t\u\v\w\x\y\z
%   Digits        \0\1\2\3\4\5\6\7\8\9
%   Exclamation   \!     Double quote  \"     Hash (number) \#
%   Dollar        \$     Percent       \%     Ampersand     \&
%   Acute accent  \'     Left paren    \(     Right paren   \)
%   Asterisk      \*     Plus          \+     Comma         \,
%   Minus         \-     Point         \.     Solidus       \/
%   Colon         \:     Semicolon     \;     Less than     \<
%   Equals        \=     Greater than  \>     Question mark \?
%   Commercial at \@     Left bracket  \[     Backslash     \\
%   Right bracket \]     Circumflex    \^     Underscore    \_
%   Grave accent  \`     Left brace    \{     Vertical bar  \|
%   Right brace   \}     Tilde         \~}
%
% \changes{v0.1}{2007/01/16}{Initial version.}
% \changes{v0.2}{2007/01/17}{new convenience commands, LPPL, bugfixes: missing babel package, ifthen-placement, loc, author markup}
% \changes{v0.3}{2007/01/22}{english documentation, replaced command changed with command replaced}
% \changes{v0.4}{2007/01/24}{pdfcolmk for improved markup, introduced author-ids, first CTAN version}
% \changes{v0.5}{2007/08/26}{reimplementation without array package, UTF-8, grayed text, change pf command arguments}
% \changes{v0.5.1}{2007/08/27}{deleted text is striked out again using package ulem, greying didn't work}
% \changes{v0.5.2}{2007/10/10}{package options for pdfcolmk, ulem, and xcolor}
% \changes{v0.5.3}{2010/11/22}{use class options (final, draft) as well}
% \changes{v0.5.4}{2011/04/25}{extract user documentation; default language changed to English; script for removal of commands}
% \changes{v0.6.0}{2011/10/19}{redefined user interface for setting options, markup, authors}
% \GetFileInfo{changes.dtx}
% \RecordChanges
%
%^^A --------------------------------------------------------------------------
%
% \maketitle
%
% \tableofcontents
% \cleardoublepage
%
% \ifENGLISH
% 	%^^A ---- introduction
\section{Introduction}

This package provides means for manual change markup.

Any comments, thoughts or improvements are welcome.
The package is maintained at \emph{sourceforge}, please see

\url{http://changes.sourceforge.net/}

for source code access, bug and feature tracker, forum etc.
If you want to contact me directly, please send an email to \href{mailto:ekleinod@edgesoft.de}{ekleinod@edgesoft.de}.
Please start your email subject with \texttt{[changes]}.

\begin{quote}
	\small\textsc{README:}
	The changes-package allows the user to manually markup changes of text, such as additions, deletions, or replacements.
	Changed text is shown in a different color; deleted text is striked out.
	The package allows free definition of additional authors and their associated color.
	It also allows you to change the markup of changes, authors, or annotations.
\end{quote}


%^^A ---- usage
\section{Using the \chpackage{changes}-package}
\label{sec:usage}

In this section a typical use case of the \chpackage{changes}-package is described.
You can find the detailed description of the package options and new commands in \autoref{sec:user}.

We start with the text you want to change.
You want to markup the changes for each author individually.
Such a change markup is well-known in WYSIWYG text processors such as \emph{LibreOffice}, \emph{OpenOffice}, or \emph{Word}.

The \chpackage{changes}-package was developed in order to support such change markup.
The package provides commands for defining authors, and for marking text as added, deleted, or replaced.
In order to use the package, you have to follow these steps:

\begin{enumerate}
	\item use \chpackage{changes}-package
	\item define authors
	\item markup text changes
	\item typeset the document with \LaTeX
	\item output list of changes
	\item remove markup
\end{enumerate}


\minisec{use \chpackage{changes}-package}

In order to activate change management, use the \chpackage{changes}-package as follows:

\chcommand{usepackage\{changes\}}

respectively

\chcommand{usepackage[<options>]\{changes\}}

You can use the options for defining the layout of the change markup.
You can change the layout after using the \chpackage{changes}-package as well.

For detailed information please refer to \autoref{sec:user:options} and \autoref{sec:user:customizingoutput}.


\minisec{define authors}

The \chpackage{changes}-package provides a default anonymous author.
If you want to track your changes depending on the author, you have to define the needed authors as follows:

\chcommand{definechangesauthor[<options>]\{id\}}

Every author is uniquely identified through his or her id.
You can give every author an optional name and/or color.

For detailed information please refer to \autoref{sec:user:authormanagement}.


\minisec{markup text changes}

Now everything is set to markup the changed text.
Please use the following commands according to your change:

for newly added text:\\
\chcommand{added[id=<id>, remark=<remark>]\{text\}}

for deleted text:\\
\chcommand{deleted[id=<id>, remark=<remark>]\{text\}}

for replaced text:\\
\chcommand{replaced[id=<id>, remark=<remark>]\{text\}}

Stating the author's id and/or a remark is optional.

For detailed information please refer to \autoref{sec:user:changemanagement}.


\minisec{typeset the document with \LaTeX}

After marking your changes in the text you are able to display them in the generated document by processing it as usual with \LaTeX.
By processing your document the changed text is layouted as you stated by the corresponding options and/or special commands.

\minisec{output list of changes}

You can print a list of changes using:

\chcommand{listofchanges[style=<list|summary>]}

The list is meant to be the analogon to the list of tables, or the list of figures.

Stating the style is optional, default is \choption{style=list}.
In order to print a quick overview of the number and kind of changes of every author, use the option \choption{style=summary}.

By running \LaTeX\ the data of the list is written into an auxiliary file.
This data is used in the next \LaTeX\ run for typesetting the list of changes.
Therefore, two \LaTeX\ runs are needed after every change in order to typeset an up-to-date list of changes.


\minisec{remove markup}

Often you want to remove the change markup after acknowledging or rejecting the changes.
You can suppress the output of changes with:

\chcommand{usepackage[final]\{changes\}}

\subsection{Available scripts}

In order to remove the markup from the \LaTeX\ source code  you can use a script from Silvano Chiaradonna.
You find the script in the directory:

\texttt{<texpath>/scripts/changes/}

The script removes all markups.
You can select or deselect markup from removal using the interactive mode.
Switch on the interactive mode with the script parameter \texttt{-i}.

%^^A ---- limitations
\section{Limitations and possible enhancements}
\label{sec:limitations}

The \chpackage{changes}-package was carefully programmed and tested.
Yet the possibility of errors in the package exists, you might encounter problem during use, or you might miss functionionality.
In that case, please go to

\url{http://changes.sourceforge.net/}

There you can report errors, ask for help in the forum, or give advice to other users.
You can view the source code, and change it according to your needs.
I will try to include your changes in the maintained package.
If you are a registered \emph{sourceforge} user you can be a co-author of the \chpackage{changes}-package.

You can write me an email too, please send it to \href{mailto:ekleinod@edgesoft.de}{ekleinod@edgesoft.de}.
In that case, please start your email subject with \texttt{[changes]}.

Change markup of texts works well, it is possible to markup whole paragraphs.
You can markup more than one paragraph at a time but occasionally this leads to errors.
You cannot markup figures or tables.

You can try putting such text in an extra file and include in with \texttt{input}.
This works sometimes, give it a try.
Kudos to Charly Arenz for this tip.

There is a problem of typesetting footnotes in special environments, such as tables or tabbings.
Since footnotes are the default markup of remarks, this would be a problem.
You can solve this problem by defining another annotation of remarks.

There are several possibilities of enhancing the \chpackage{changes}-package.
I will describe but a few here, I will not implement them due to lack of time and/or skill.
You can have a look at the more complete list of enhancements on the \emph{sourceforge} page.

\begin{itemize}
	\item selecting of acknowledged and rejected texts; deletion of the corresponding markup
	\item markup of more than one paragraph
	\item markup of figures and tables
	\item automatic markup based on diff information (with regard to the limitations, such as markup of paragraphs, figures etc.)
	\item translation of language dependent texts and the user documentation in other languages
\end{itemize}


%^^A ---- user interface
\section{User interface of the \chpackage{changes}-package}
\label{sec:user}

This section describes the user interface of the \chpackage{changes}-package, i.e. all options and commands of the package.
Every option respectively new command is described.
If you want to see the options and commands in action, please refer to the examples in

\texttt{<texpath>/doc/latex/changes/examples/}

The example files are named with the used option respectively command.

A preliminary remark regarding typesetting of replaced text: replaced text is always typeset as follows: \meta{new text}\meta{old text}.
Thus, there is no possiblity to influence the output of replaced text directly, but via changing the output of added respectively deleted text.


%^^A -- options
\subsection{Package Options}
\label{sec:user:options}

\subsubsection{draft}

The \choption{draft}-option enables markup of changes.
The list of changes is available via \chcommand{listofchanges}.
This option is the default option, if no other option is selected.

The \chpackage{changes} package reuses the declaration of \choption{draft} in \chcommand{documentclass}.
The local declaration of \choption{final} overrules the declaration of \choption{draft} in \chcommand{documentclass}.

\chcommand{usepackage[draft]\{changes\}} \Corresponds\ \chcommand{usepackage\{changes\}}

\subsubsection{final}
The \choption{final}-option disables markup of changes, only the correct text will be shown.
The list of changes is disabled, too.

The \chpackage{changes} package reuses the declaration of \choption{final} in \chcommand{documentclass}.
The local declaration of \choption{draft} overrules the declaration of \choption{final} in \chcommand{documentclass}.

\chcommand{usepackage[final]\{changes\}}


\subsubsection{markup}

The \choption{markup} option chooses a predefined visual markup of changed text.
The default markup is chosen if no explicit markup is given.
The markup chosen with \choption{markup} can be overwritten with the more special markup options \choption{addedmarkup} and/or \choption{deletedmarkup}.

The following values are allowed:
\begin{description}
	\item [\choption{default}] colored markup of added text, striked out for deleted text (default markup)
	\item [\choption{underlined}] underlined for added text, striked out for deleted text
	\item [\choption{bfit}] bold added text, italic deleted text
	\item [\choption{nocolor}] no colored markup, underlined for added text, striked out for deleted text
\end{description}

\begin{chusage}
		\>\chcommand{usepackage[markup=\meta{markup}]\{changes\}}\\
	\usageexample
		\>\chcommand{usepackage[markup=default]\{changes\}} \Corresponds\ \chcommand{usepackage\{changes\}}\\
		\>\chcommand{usepackage[markup=underlined]\{changes\}}\\
		\>\chcommand{usepackage[markup=bfit]\{changes\}}\\
		\>\chcommand{usepackage[markup=nocolor]\{changes\}}
\end{chusage}


\subsubsection{addedmarkup, deletedmarkup}

The \choption{addedmarkup} option chooses a predefined visual markup of added text.
The \choption{deletedmarkup} option chooses a predefined visual markup of deleted text respectively.
The default markup is chosen if no explicit markup is given.
The options \choption{addedmarkup} and \choption{deletedmarkup} overwrite the markup chosen with \choption{markup}.

The following values are allowed:
\begin{description}
	\item [\choption{none}] no markup -- example (default markup for added text)
	\item [\choption{uline}] underlined text -- \uline{example}
	\item [\choption{uuline}] double underlined text -- \uuline{example}
	\item [\choption{uwave}] wavy underlined text -- \uwave{example}
	\item [\choption{dashuline}] dashed underlined text -- \dashuline{example}
	\item [\choption{dotuline}] dotted underlined text -- \dotuline{example}
	\item [\choption{sout}] striked out text -- \sout{example} (default markup for deleted text)
	\item [\choption{xout}] crossed out text -- \xout{example}
	\item [\choption{bf}] bold text -- \textbf{example}
	\item [\choption{it}] italic text -- \textit{example}
	\item [\choption{sl}] slanted text -- \textsl{example}
	\item [\choption{em}] emphasized text -- \emph{example}
\end{description}

\begin{chusage}
		\>\chcommand{usepackage[addedmarkup=\meta{markup}]\{changes\}}\\
	\usageexample
		\>\chcommand{usepackage[addedmarkup=none]\{changes\}} \Corresponds\ \chcommand{usepackage\{changes\}}\\
		\>\chcommand{usepackage[addedmarkup=uline]\{changes\}}\\
\end{chusage}

\begin{chusage}
		\>\chcommand{usepackage[deletedmarkup=\meta{markup}]\{changes\}}\\
	\usageexample
		\>\chcommand{usepackage[deletedmarkup=sout]\{changes\}} \Corresponds\ \chcommand{usepackage\{changes\}}\\
		\>\chcommand{usepackage[deletedmarkup=xout]\{changes\}}\\
		\>\chcommand{usepackage[deletedmarkup=uwave]\{changes\}}
\end{chusage}



\subsubsection{authormarkup}

The \choption{authormarkup} option chooses a predefined visual markup of the author's identification.
The default markup is chosen if no explicit markup is given.

The following values are allowed:
\begin{description}
	\item [\choption{superscript}] superscripted text -- text\textsuperscript{author} (default markup)
	\item [\choption{subscript}] subscripted text -- text\textsubscript{author}
	\item [\choption{brackets}] text in brackets -- text(author)
	\item [\choption{footnote}] text in footnote -- text\footnote{author}
	\item [\choption{none}] no author identification
\end{description}

\begin{chusage}
		\>\chcommand{usepackage[authormarkup=\meta{markup}]\{changes\}}\\
	\usageexample
		\>\chcommand{usepackage[authormarkup=superscript]\{changes\}} \Corresponds\ \chcommand{usepackage\{changes\}}\\
		\>\chcommand{usepackage[authormarkup=subscript]\{changes\}}\\
		\>\chcommand{usepackage[authormarkup=brackets]\{changes\}}\\
		\>\chcommand{usepackage[authormarkup=footnote]\{changes\}}\\
		\>\chcommand{usepackage[authormarkup=none]\{changes\}}
\end{chusage}



\subsubsection{authormarkupposition}

The \choption{authormarkupposition} option chooses the position of the author's identification.
The default value is chosen if no explicit markup is given.

The following values are allowed:
\begin{description}
	\item [\choption{right}] right of the text -- text\textsuperscript{example} (default value)
	\item [\choption{left}] left of the text -- \textsuperscript{example}text
\end{description}

\begin{chusage}
		\>\chcommand{usepackage[authormarkupposition=\meta{markup}]\{changes\}}\\
	\usageexample
		\>\chcommand{usepackage[authormarkupposition=right]\{changes\}} \Corresponds\ \chcommand{usepackage\{changes\}}\\
		\>\chcommand{usepackage[authormarkupposition=left]\{changes\}}
\end{chusage}



\subsubsection{authormarkuptext}

The \choption{authormarkuptext} option chooses the text that is used for the author's identification.
The default value is chosen if no explicit markup is given.

The following values are allowed:
\begin{description}
	\item [\choption{id}] author's id -- text\textsuperscript{id} (default value)
	\item [\choption{name}] author's name -- text\textsuperscript{authorname}
\end{description}

\begin{chusage}
		\>\chcommand{usepackage[authormarkuptext=\meta{markup}]\{changes\}}\\
	\usageexample
		\>\chcommand{usepackage[authormarkuptext=id]\{changes\}} \Corresponds\ \chcommand{usepackage\{changes\}}\\
		\>\chcommand{usepackage[authormarkuptext=name]\{changes\}}
\end{chusage}



\subsubsection{ulem}

All options for the \chpackage{ulem} package can be specified as parameters of the \choption{ulem}-option.
Two or more options have to be put in curly brackets.

\begin{chusage}
		\>\chcommand{usepackage[ulem=\meta{options}]\{changes\}}\\
	\usageexample
		\>\chcommand{usepackage[ulem=normalem]\{changes\}}\\
		\>\chcommand{usepackage[ulem=\{normalem,normalbf\}]\{changes\}}
\end{chusage}



\subsubsection{xcolor}

All options for the \chpackage{xcolor} package can be specified as parameters of the \choption{xcolor}-option.
Two or more option have to be embraced in curly brackets.

\begin{chusage}
		\>\chcommand{usepackage[xcolor=\meta{options}]\{changes\}}\\
	\usageexample
		\>\chcommand{usepackage[xcolor=dvipdf]\{changes\}}\\
		\>\chcommand{usepackage[xcolor=\{dvipdf,gray\}]\{changes\}}
\end{chusage}



%^^A -- Change management ----------------------------------------------------------
\subsection{Change management}
\label{sec:user:changemanagement}

\subsubsection{\chcommand{added}}
\DescribeMacro{\added}

The command \chcommand{added} marks new text.
The new text is the mandatory argument for the command, thus it is written in curly braces.
The optional argument contains key-value-pairs for author-id and remark.
The author-id has to be defined using \chcommand{definechangesauthor}.
If the remark contains special characters or spaces, use curly brackets to enclose the remark.

\begin{chusage}
		\>\chcommand{added[id=\meta{author's id}, remark=\meta{remark}]\{\meta{new text}\}}\\
	\usageexample
		\>\texttt{This is \chcommand{added}[id=EK]\{new\} text.}\\
		\>This is \added[id=EK]{new} text.\\
		\>\texttt{This is \chcommand{added}[id=EK, remark=\{has to be in it\}]\{new\} text.}\\
		\>This is \added[id=EK, remark={has to be in it}]{new} text.\\
		\>\texttt{This is \chcommand{added}[remark=anonymous]\{new\} text.}\\
		\>This is \added[remark=anonymous]{new} text.
\end{chusage}


\subsubsection{\chcommand{deleted}}
\DescribeMacro{\deleted}

The command \chcommand{deleted} marks deleted text.
For arguments see \chcommand{added}.

\begin{chusage}
		\>\chcommand{deleted[id=\meta{author's id}, remark=\meta{remark}]\{\meta{deleted text}\}}\\
	\usageexample
		\>\texttt{This is \chcommand{deleted}[remark=obsolete]\{bad\} text.}\\
		\>This is \deleted[remark=obsolete]{bad} text.
\end{chusage}


\subsubsection{\chcommand{replaced}}
\DescribeMacro{\replaced}

The command \chcommand{replaced} marks replaced text.
Mandatory arguments are the new text and the old text.
For optional arguments see \chcommand{added}.

\begin{chusage}
		\>\chcommand{replaced[id=\meta{author's id}, remark=\meta{remark}]\{\meta{new text}\}\{\meta{old text}\}}\\
	\usageexample
		\>\texttt{This is \chcommand{replaced}[id=EK]\{nice\}\{bad\} text.}\\
		\>This is \replaced[id=EK]{nice}{bad} text.
\end{chusage}



\subsubsection{\chcommand{listofchanges}}
\DescribeMacro{\listofchanges}

The command \chcommand{listofchanges} outputs a list or summary of changes.
The first \LaTeX-run creates an auxiliary file, the second run uses the data of this file.
Therefore you need two \LaTeX-runs for an up-to-date list of changes.

The\marginpar{new since v2.0.0} style argument is optional, by default the list of changes is printed.
If you want to print a summary you have to use the option \choption{style=summary}.

\begin{chusage}
		\>\chcommand{listofchanges[style=<list|summary>]}
\end{chusage}


%^^A -- Author management -----------------------------------------------------
\subsection{Author management}
\label{sec:user:authormanagement}

\subsubsection{\chcommand{definechangesauthor}}
\DescribeMacro{\definechangesauthor}

The command \chcommand{definechangesauthor} defines a new author for changes.
You have to define a unique author's id, special characters or spaces are not allowed within the author's id.
You may define a corresponding color and the author's name.
If you do not define a color, black is used.
The author's name is used in the list of changes and in the markup, if you set the corresponding option.

\begin{chusage}
		\>\chcommand{definechangesauthor[name=\{\meta{author's name}\}, color=\{\meta{color}\}]\{\meta{author's id}\}}\\
	\usageexample
		\>\chcommand{definechangesauthor\{EK\}}\\
		\>\chcommand{definechangesauthor[color=orange]\{EK\}}\\
		\>\chcommand{definechangesauthor[name=\{Ekkart Kleinod\}]\{EK\}}\\
		\>\chcommand{definechangesauthor[name=\{Ekkart Kleinod\}, color=orange]\{EK\}}
\end{chusage}


%^^A -- Adaptation of the output -----------------------------------------------------
\subsection{Adaptation of the output}
\label{sec:user:customizingoutput}


\subsubsection{\chcommand{setaddedmarkup}}
\DescribeMacro{\setaddedmarkup}

The command \chcommand{setaddedmarkup} defines the layout of added text.
The default markup is colored text, or the markup set with the option \choption{markup} respectively \choption{addedmarkup}.

Values for definition: any \LaTeX-commands, added text can be used with ``\#1''.

\begin{chusage}
		\>\chcommand{setaddedmarkup\{\meta{definition}\}}\\
	\usageexample
		\>\chcommand{setaddedmarkup\{}\chcommand{emph\{\#1\}}\}\\
		\>\chcommand{setaddedmarkup\{+++: \#1\}}
\end{chusage}


\subsubsection{\chcommand{setdeletedmarkup}}
\DescribeMacro{\setdeletedmarkup}

The command \chcommand{setdeletedmarkup} defines the layout of deleted text.
The default markup is striked-out, or the markup set with the option \choption{markup} respectively \choption{deletedmarkup}.

Values for definition: any \LaTeX-commands, deleted0 text can be used with ``\#1''.

\begin{chusage}
		\>\chcommand{setdeletedmarkup\{\meta{definition}\}}\\
	\usageexample
		\>\chcommand{setdeletedmarkup\{}\chcommand{emph\{\#1\}}\}\\
		\>\chcommand{setdeletedmarkup\{---: \#1\}}
\end{chusage}


\subsubsection{\chcommand{setauthormarkup}}
\DescribeMacro{\setauthormarkup}

The command \chcommand{setauthormarkup} defines the layout of the author's markup in the text.
The default markup is a superscripted author's text.

Values for definition: any \LaTeX-commands, author's text can be used with ``\#1''.

\begin{chusage}
		\>\chcommand{setauthormarkup\{\meta{definition}\}}\\
	\usageexample
		\>\chcommand{setauthormarkup\{(\#1)\}}\\
		\>\chcommand{setauthormarkup\{(\#1)\textasciitilde{}-{}-\textasciitilde{}\}}\\
		\>\chcommand{setauthormarkup\{}\chcommand{marginpar\{\#1\}\}}
\end{chusage}


\subsubsection{\chcommand{setauthormarkupposition}}
\DescribeMacro{\setauthormarkupposition}

The command \chcommand{setauthormarkupposition} defines the position of the author's markup relative to the changed text.
The default position is right of the changed text.

Possible values: \emph{left} == left of the changes; all other values: right

\begin{chusage}
		\>\chcommand{setauthormarkupposition\{\meta{position}\}}\\
	\usageexample
		\>\chcommand{setauthormarkupposition\{left\}}
\end{chusage}



\subsubsection{\chcommand{setauthormarkuptext}}
\DescribeMacro{\setauthormarkuptext}

The command \chcommand{setauthormarkuptext} defines the text for the author's markup.
The default markup is the author's id.

Possible values: \emph{name} == author's name; all other values: author's id

\begin{chusage}
		\>\chcommand{setauthormarkuptext\{\meta{text}\}}\\
	\usageexample
		\>\chcommand{setauthormarkuptext\{name\}}
\end{chusage}



\subsubsection{\chcommand{setremarkmarkup}}
\DescribeMacro{\setremarkmarkup}

The command \chcommand{setremarkmarkup} defines the layout of remarks in the text.
The default markup typesets the remark in a footnote.

Values for definition: any \LaTeX-commands, author's id can be used with ``\#1'', the remark can be shown using ``\#2''.
Using the author's id you can use the author's color with \texttt{Changes@Color\#1}.

\begin{chusage}
		\>\chcommand{setremarkmarkup\{\meta{definition}\}}\\
	\usageexample
		\>\chcommand{setremarkmarkup\{(\#2 --- \#1)\}}\\
		\>\chcommand{setremarkmarkup\{\chcommand{footnote}\{\#1:\chcommand{textcolor\{Changes@Color\#1\}}\{\#2\}\}\}}
\end{chusage}



\subsubsection{\chcommand{setsocextension}}
\DescribeMacro{\setsocextension}

The\marginpar{new since v2.0.0} command \chcommand{setsocextension} sets the extension of the auxiliary file for the summary of changes (soc-file\footnote{%
	``soc'' stands for ``summary of changes''.
}).
The default extension is ``\texttt{soc}''.
In the example stated below, the soc-file for ``\texttt{foo.tex}'' would be named ``\texttt{foo.changes}'' instead of the default name ``\texttt{foo.soc}''.

\begin{chusage}
		\>\chcommand{setsocextension\{\meta{extension}\}}\\
	\usageexample
		\>\chcommand{setsocextension\{changes\}}
\end{chusage}



%^^A ---- other
\subsection{Other new commands}
\label{sec:user:other}

\subsubsection{\chcommand{textsubscript}}
\DescribeMacro{\textsubscript}

\LaTeX\ provides the command \chcommand{textsuperscript}, but not it's counterpart \chcommand{textsubscript}.
If the command is not defined yet, it will be provided by the \chpackage{changes}-package.
If the command is defined yet, it will not be changed.
\begin{chusage}
		\>\chcommand{textsubscript\{\meta{text}\}}\\
	\usageexample
		\>\texttt{This is a \chcommand{textsubscript\{subscript\}} text.}\\
		\>This is a \textsubscript{subscript} text.
\end{chusage}


%^^A -- packages
\subsection{Used packages}
\label{sec:user:packages}

The \chpackage{changes}-package uses already existing packages for it's functions.
You will find detailed description of the packages in their distributions.

The following packages are always required and have to be installed for the \chpackage{changes}-package:
\begin{description}
	\item [xifthen] provides an enhanced \chcommand{if}-command as well as a \texttt{while}-loop
	\item [xkeyval] provides options with key-value-pairs
\end{description}

The following packages are sometimes required and have to be installed if used by the corresponding option:
\begin{description}
	\item [pdfcolmk] loaded if colored text is used for markup (default markup); solves the problem of colored text and page breaks (with pdflatex)
	\item [ulem] loaded if text has to be striked or exed out (default markup)
	\item [xcolor] loaded if colored text is used for markup (default markup)
\end{description}


%^^A -- Authors -------------------------------------------------------------
\section{Authors}
\label{sec:authors}

Several authors contributed to the \chpackage{changes}-package.
The authors are (in alphabetical order):
\begin{itemize}
	\item Chiaradonna, Silvano
	\item Giovannini, Daniele
	\item Kleinod, Ekkart
	\item Wölfel, Philipp
	\item Wolter, Steve
\end{itemize}



%^^A -- Versions -------------------------------------------------------------
\section{Versions}
\label{sec:versions}

\minisec{Version 2.0.1}

Date: 2013/08/10
\begin{itemize}
	\item no changes in code or behavior
	\item fixed upload problems with CTAN (wrong line endings)
	\item put all needed files in CTAN archive
\end{itemize}

\minisec{Version 2.0.0}

Date: 2013/06/30
\begin{itemize}
	\item ``real'' list of changes, old summary now with optional parameter \choption{style=summary}
	\item fixed problem with special characters in summary of changes
	\item renamed \chcommand{setlocextension} to \chcommand{setsocextension}
	\item new author markup \choption{none}
	\item completed script description with \texttt{-i} parameter
\end{itemize}

\minisec{Version 1.0.0}

Date: 2012/04/25
\begin{itemize}
	\item key-value-interface for change commands
	\item fixed bug (crash) with special characters in list of changes
	\item added space before author name in list of changes
	\item error message if an unknown author id is used
\end{itemize}

\minisec{Version 0.6.0}

Date: 2012/01/11
\begin{itemize}
	\item Italian translations of captions by Daniele Giovannini
	\item redefined user interface for setting options and definitions of markup and authors
	\item restructuring and code improvement
	\item improved documentation including typical use case
	\item example files for all options and commands
	\item by default remarks are not colored anymore
\end{itemize}

\minisec{Version 0.5.4}

Date: 2011/04/25
\begin{itemize}
	\item extraction of user documentation in separate file
	\item default language changed to English
	\item new script for removal of \chpackage{changes} commands by Silvano Chiaradonna
\end{itemize}

\minisec{Version 0.5.3}

Date: 2010/11/22
\begin{itemize}
	\item document options of \chcommand{documentclass} are used too (suggestion and code of Steve Wolter)
\end{itemize}

\minisec{Version 0.5.2}

Date: 2007/10/10
\begin{itemize}
	\item package options for \chpackage{ulem} and \chpackage{xcolor} are passed to the packages
\end{itemize}

\minisec{Version 0.5.1}

Date: 2007/08/27
\begin{itemize}
	\item deleted text is striked out again using package \chpackage{ulem}, greying didn't work
\end{itemize}

\minisec{Version 0.5}

Date: 2007/08/26
\begin{itemize}
	\item no usage of package \chpackage{arrayjob} anymore, thus no errors using package \chpackage{array}
	\item switch to UTF-8-encoding
	\item no usage of package \chpackage{soul} anymore, thus no errors using UTF-8-encoding
	\item markup for deleted text changed to gray background, because there's no possibility to conveniently strike out UTF-8-text
	\item new optional argument for author's name
	\item colored list of changes
	\item changed loc file format
	\item improved English documentation
\end{itemize}

\minisec{Version 0.4}

Date: 2007/01/24
\begin{itemize}
	\item included \chpackage{pdfcolmk} to solve problem with colored text and page breaks
	\item extended \chcommand{setremarkmarkup} with author's id for using color in remarks
	\item by default remarks are colored now
	\item first version uploaded to CTAN
\end{itemize}

\minisec{Version 0.3}

Date: 2007/01/22
\begin{itemize}
	\item English user-documentation
	\item replaced command \chcommand{changed} with \chcommand{replaced}
	\item improved \choption{final}-option: no additional space
\end{itemize}

\minisec{Version 0.2}

Date: 2007/01/17
\begin{itemize}
	\item defined loc-names when missing \chpackage{babel}-package
	\item new commands \chcommand{setauthormarkup}, \chcommand{setlocextension}, \chcommand{setremarkmarkup}
	\item generated examples
	\item inserted LPPL
\end{itemize}
Bugfixes
\begin{itemize}
	\item fixed wrong \chpackage{ifthen} package placement
	\item fixed error in loc, always showing ``added''
	\item fixed authormarkup (\chcommand{if}-condition not bugfree)
\end{itemize}

\minisec{Version 0.1}

Date: 2007/01/16
\begin{itemize}
	\item initial version
	\item commands \chcommand{added}, \chcommand{deleted}, and \chcommand{changed}
\end{itemize}


%^^A ---- copyright, license
\section{Distribution, Copyright, License}

Copyright 2007-2013 Ekkart Kleinod (\href{mailto:ekleinod@edgesoft.de}{ekleinod@edgesoft.de})

This work may be distributed and/or modified under the conditions of the \LaTeX\ Project Public License, either version~1.3 of this license or any later version.
The latest version of this license is in \url{http://www.latex-project.org/lppl.txt} and version~1.3 or later is part of all distributions of \LaTeX\ version 2005/12/01 or later.

This work has the LPPL maintenance status ``maintained''.
The current maintainer of this work is Ekkart Kleinod.

This work consists of the files

\begin{tabbing}
	mm\=\kill
	\>\texttt{source/latex/changes/changes.drv}\\
	\>\texttt{source/latex/changes/changes.dtx}\\
	\>\texttt{source/latex/changes/changes.ins}\\
	\>\texttt{source/latex/changes/examples.dtx}\\
	\>\texttt{source/latex/changes/README}\\
	\>\texttt{source/latex/changes/userdoc/*.tex}\\

	\>\texttt{scripts/changes/delcmdchanges.bash}
\end{tabbing}


and the derived files

\begin{tabbing}
	mm\=\kill
	\>\texttt{doc/latex/changes/changes.english.pdf}\\
	\>\texttt{doc/latex/changes/changes.english.withcode.pdf}\\
	\>\texttt{doc/latex/changes/changes.ngerman.pdf}\\

	\>\texttt{doc/latex/changes/examples/changes.example.*.tex}\\
	\>\texttt{doc/latex/changes/examples/changes.example.*.pdf}\\

	\>\texttt{tex/latex/changes/changes.sty}
\end{tabbing}


%^^A end of user documentation


% \fi
% \ifGERMAN
% 	%^^A ---- introduction
\section{Einleitung}

Dieses Paket dient dazu, manuelle Änderungsmarkierung anzubieten.

Verbesserungsvorschläge, Gedanken oder Kritik sind willkommen.
Das Paket wird auf sourceforge gehalten, bitte gehen Sie zu

\url{http://changes.sourceforge.net/}

für Quellcodezugang, Fehler- und Featuretracker, Forum etc.
Wenn Sie mich direkt kontaktieren wollen, mailen Sie bitte an \href{mailto:ekleinod@edgesoft.de}{ekleinod@edgesoft.de}, bitte starten Sie Ihr Mail-Subject mit \texttt{[changes]}.

\begin{quote}
	\small\textsc{README:}
	Das changes-Paket dient zur manuellen Markierung von geändertem Text, insbesondere Einfügungen, Löschungen und Ersetzungen.
	Der geänderte Text wird farbig markiert und, bei gelöschtem Text, durchgestrichen.
	Das Paket ermöglicht die freie Definition von Autoren und deren zugeordneten Farben.
	Es erlaubt zusätzlich die Definition des Autor- und Anmerkungsmarkups.
\end{quote}

%^^A ---- usage
\section{Benutzung des \chpackage{changes}-Pakets}
\label{sec:usage}

In diesem Kapitel wird die typische Nutzung des \chpackage{changes}-Pakets beschrieben.
Dabei wird ein typischer Anwendungsfall geschildert.
Die ausführliche Beschreibung der Paketoptionen und neuen Befehle finden Sie nicht hier, sondern in \autoref{sec:user}.

Ausgangslage ist ein Text, an dem Änderungen vorgenommen werden sollen.
Diese Änderungen sollen markiert werden, und zwar für jeden Autor einzeln.
Eine solche Änderungsmarkierung ist z.\,B.\ von WYSIWYG-Textprogrammen wie \emph{LibreOffice}, \emph{OpenOffice} oder \emph{Word} bekannt.

Zu diesem Zweck wurde das \chpackage{changes}-Paket entwickelt.
Das Paket stellt Befehle zur Verfügung, um verschiedene Autoren zu definieren und Text als zugefügt, gelöscht oder geändert zu markieren.
Um das Paket zu nutzen, müssen Sie folgende Schritte ausführen:
\begin{enumerate}
	\item \chpackage{changes}-Paket einbinden
	\item Autoren definieren
	\item Textänderungen markieren
	\item Dokument mit \LaTeX\ setzen
	\item für die Liste von Änderungen Dokument zweimal mit \LaTeX\ setzen
	\item Markierungen entfernen
\end{enumerate}

\minisec{\chpackage{changes}-Paket einbinden}

Um die Änderungsverfolgung zu aktivieren, ist das \chpackage{changes}-Paket wie folgt einzubinden:

\chcommand{usepackage\{changes\}}

bzw.

\chcommand{usepackage[<optionen>]\{changes\}}

Mit den verfügbaren Optionen bestimmen Sie das Aussehen der Änderungsmarkierungen.
Sie können das Aussehen der Änderungsmarkierungen auch nach Einbinden des \chpackage{changes}-Paket verändern.

Für Details lesen Sie bitte \autoref{sec:user:options} und \autoref{sec:user:customizingoutput}.

\minisec{Autoren definieren}

Das \chpackage{changes}-Paket stellt einen vordefinierten anonymen Autor zur Verfügung.
Wenn Sie jedoch die Änderungen per AutorIn verfolgen wollen, müssen Sie die entsprechenden Autoren definieren.
Dies geht wie folgt:

\chcommand{definechangesauthor[<optionen>]\{ID\}}

Über die ID werden der/die AutorIn und die zugehörigen Textänderungen eindeutig identifiziert.
Optional können Sie einen Namen angeben und dem/der AutorIn eine eigene Farbe zuweisen.

Für Details lesen Sie bitte \autoref{sec:user:authormanagement}.

\minisec{Textänderungen markieren}

Jetzt ist alles vorbereitet, um den geänderten Text zu markieren.
Benutzen Sie bitte je nach Änderung die folgenden Befehle:

für neu zugefügten Text:\\
\chcommand{added[<ID>][<Anmerkung>]\{Text\}}

für gelöschten Text:\\
\chcommand{deleted[<ID>][<Anmerkung>]\{Text\}}

für geänderten Text:\\
\chcommand{replaced[<ID>][<Anmerkung>]\{neuer Text\}\{alter Text\}}

Die Angabe von Autoren-ID und einer Anmerkung ist optional.

Für Details lesen Sie bitte \autoref{sec:user:changemanagement}.

\minisec{Dokument mit \LaTeX\ setzen}

Nachdem Sie die Änderungen im \LaTeX-Text markiert haben, können Sie sie im erzeugten Dokument sichtbar machen, indem Sie das Dokument ganz normal übersetzen.
Durch die Übersetzung wird der geänderte Text so markiert, wie Sie das mittels der Optiuonen bzw.\ speziellen Befehle eingestellt haben.

\minisec{für die Liste von Änderungen Dokument zweimal mit \LaTeX\ setzen}

Sie können sich eine Liste der Änderungen ausgeben lassen.
Dies erfolgt mit dem Kommando:

\chcommand{listofchanges}

Die Ausgabe ist gedacht als Analogon zur Liste von Tabellen oder Abbildungen.
Sie dient dazu, einen schnellen Überblick über Art und Anzahl der Änderungen abhängig von dem/der AutorIn zu bekommen.

Bei jedem \LaTeX-Lauf werden die Daten für diese Liste in eine Hilfsdatei geschrieben.
Beim nächsten \LaTeX-Lauf wewrden dann diese Daten genutzt, um die Änderungsliste anzuzeigen.
Daher sind nach jeder Änderung zwei \LaTeX-Läufe notwendig, um eine aktuelle Änderungsliste anzuzeigen.

\minisec{Markierungen entfernen}

Oft ist es der Fall, dass die Änderungen eines Dokuments angenommen oder abgelehnt werden und nach diesem Prozess die Änderungsmarkierungen entfernt werden sollen.
Sie können die Ausgabe der Änderungsmarkierungen per Option beim Einbinden des \chpackage{changes}-Pakets unterdrücken:

\chcommand{usepackage[final]\{changes\}}

Für die Entfernung der Markierungen aus dem Quelltext steht ein Script von Silvano Chiaradonna zur Verfügung.
Das Script liegt im Verzeichnis:

\texttt{<texpfad>/scripts/changes/}

Das Script entfernt alle Markierungen, es ist nicht möglich, die zu entfernenden Markierungen zu selektieren.

%^^A ---- limitations
\section{Einschränkungen und Erweiterungsmöglichkeiten}
\label{sec:limitations}

Das \chpackage{changes}-Paket ist sorgfältig programmiert und getestet worden.
Dennoch kann es vorkommen, dass Fehler im Paket sind, dass die Benutzung problematisch ist oder dass eine Funktion fehlt, die Sie gerne hätten.
In diesem Fall gehen Sie bitte zu

\url{http://changes.sourceforge.net/}

Dort können Sie Fehler melden, im Forum um Hilfe fragen oder Tips einstellen.
Sie können dort auch den Quellcode ansehen und nach Ihren Wünschen ändern bzw.\ erweitern.
Ich werde mich dann bemühen, Ihre Änderungen einzuarbeiten.
Sie können auch als Co-Autor am Paket mitarbeiten, wenn Sie bei \emph{sourceforge} angemeldet sind.

Sie können mir auch eine Mail schreiben an \href{mailto:ekleinod@edgesoft.de}{ekleinod@edgesoft.de}, in diesem Fall starten Sie bitte Ihr Mail-Subject mit \texttt{[changes]}.

Die Änderungsmarkierung von Text funktioniert recht gut, es können auch ganze Absätze markiert werden.
Die Markierung von mehreren Absätzen gleichzeitig, von Bildern und Tabellen ist nicht möglich.

Folgende Erweiterungen fallen mir spontan ein, die ich jedoch nicht selbst programmieren werde (weil mir Zeit und oft auch die Fähigkeit fehlt):
\begin{itemize}
	\item Auswahl der anzunehmenden/abzulehnenden Änderungen mit entsprechendem Löschen des Textes
	\item Markierung von mehreren Absätzen
	\item Markierung von Bildern und Tabellen
	\item automatische Markierung anhand von diff-Informationen (unter Berücksichtigung der Einschränkungen bzgl.\ Absätzen, Bildern, etc.)
\end{itemize}

%^^A ---- user interface
\section{Die Benutzerschnittstelle des \chpackage{changes}-Pakets}
\label{sec:user}

In diesem Kapitel wird die Nutzerschnittstelle des \chpackage{changes}-Pakets vorgestellt, d.\,h.\ alle Optionen und Kommandos.
Jede Option bzw. jedes neue Kommando werden beschrieben.
Wenn Sie die Optionen und Kommandos im Beispiel sehen wollen, sehen Sie bitte in das Beispielverzeichnis unter

\texttt{<texpfad>/doc/latex/changes/examples/}

Die Beispieldateien sind mit der benutzten Option bzw. dem benutzten Kommando benannt.

%^^A -- options
\subsection{Paketoptionen}
\label{sec:user:options}

\subsubsection{draft}

Die \choption{draft}-Option bewirkt, dass alle Änderungen markiert werden.
Die Änderungsliste kann durch \chcommand{listofchanges} ausgegeben werden.
Ohne Optionsangabe wird \choption{draft} automatisch eingestellt.

Die Angabe von \choption{draft} in \chcommand{documentclass} wird vom \chpackage{changes}-Paket mitgenutzt.
Die lokale Angabe von \choption{final} überstimmt die Angabe von \choption{draft} in \chcommand{documentclass}.

\chcommand{usepackage[draft]\{changes\}}

\subsubsection{final}

Die \choption{final}-Option bewirkt, dass alle Änderungsmarkierungen ausgeblendet werden und nur noch der korrekte Text ausgegeben wird.
Die Änderungsliste wird ebenfalls unterdrückt.

Die Angabe von \choption{final} in \chcommand{documentclass} wird vom \chpackage{changes}-Paket mitgenutzt.
Die lokale Angabe von \choption{draft} überstimmt die Angabe von \choption{final} in \chcommand{documentclass}.

\chcommand{usepackage[final]\{changes\}}

\subsubsection{markup}

Die \choption{markup}-Option wählt ein vordefiniertes visuelles Markup für geänderten Text.
Das default-Markup wird gewählt, wenn die Option nicht gesetzt wird.
Das mit \choption{markup} gewählte Markup kann mit den spezielleren Optionen \choption{addedmarkup} und/oder \choption{deletedmarkup} geändert werden.

Die folgenden Werte sind erlaubt:
\begin{description}
	\item [\choption{default}] farbige Markierung von zugefügtem Text, gelöschter Text wird durchgestrichen (default-Markup)
	\item [\choption{underlined}] zugefügter Text wird unterstrichen, gelöschter Text wird durchgestrichen
	\item [\choption{bfit}] fetter zugefügter Text, schräger gelöschter Text
	\item [\choption{nocolor}] es werden keine Farben verwendet, zugefügter Text wird unterstrichen, gelöschter Text wird durchgestrichen
\end{description}

Beispiele:

\chcommand{usepackage[markup=default]\{changes\}}\\
\chcommand{usepackage[markup=underlined]\{changes\}}\\
\chcommand{usepackage[markup=bfit]\{changes\}}\\
\chcommand{usepackage[markup=nocolor]\{changes\}}

\subsubsection{addedmarkup, deletedmarkup}

Die \choption{addedmarkup}-Option wählt ein visuelles Markup für zugefügten Text.
Die \choption{deletedmarkup}-Option wählt analog ein visuelles Markup für gelöschten Text.
Das default-Markup wird gewählt, wenn die Option nicht gesetzt wird.
Die Optionen \choption{addedmarkup} und \choption{deletedmarkup} überschreiben das mit \choption{markup} gewählte Markup.

Die folgenden Werte sind erlaubt:
\begin{description}
	\item [\choption{none}] kein Markup -- Beispiel (default-Markup für zugefügten Text)
	\item [\choption{uline}] unterstrichener Text -- \uline{Beispiel}
	\item [\choption{uuline}] doppelt unterstrichener Text -- \uuline{Beispiel}
	\item [\choption{uwave}] gewellt unterstrichener Text -- \uwave{Beispiel}
	\item [\choption{dashuline}] gestrichelt unterstrichener Text -- \dashuline{Beispiel}
	\item [\choption{dotuline}] gepunktet unterstrichener Text -- \dotuline{Beispiel}
	\item [\choption{sout}] durchgestrichener Text -- \sout{Beispiel} (default-Markup für gelöschten Text)
	\item [\choption{xout}] schräg durchgestrichener Text -- \xout{Beispiel}
	\item [\choption{bf}] fetter Text -- \textbf{Beispiel}
	\item [\choption{it}] italic Text -- \textit{Beispiel}
	\item [\choption{sl}] schräger Text -- \textsl{Beispiel}
	\item [\choption{em}] hervorgehobener Text -- \emph{Beispiel}
\end{description}

Beispiele:

\chcommand{usepackage[addedmarkup=none]\{changes\}}\\
\chcommand{usepackage[addedmarkup=uline]\{changes\}}\\
\chcommand{usepackage[deletedmarkup=sout]\{changes\}}\\
\chcommand{usepackage[deletedmarkup=xout]\{changes\}}\\
\chcommand{usepackage[deletedmarkup=uwave]\{changes\}}

\subsubsection{authormarkup}

Die \choption{authormarkup}-Option wählt ein visuelles Markup für die Autor-Identifizierung.
Das default-Markup wird gewählt, wenn die Option nicht gesetzt wird.

Die folgenden Werte sind erlaubt:
\begin{description}
	\item [\choption{superscript}] hochgestellter Text -- Text\textsuperscript{Beispiel} (default-Markup)
	\item [\choption{subscript}] tiefgestellter Text -- Text\textsubscript{Beispiel}
	\item [\choption{brackets}] Text in Klammern -- Text(Beispiel)
	\item [\choption{footnote}] Text in einer Fußnote -- Text\footnote{Beispiel}
\end{description}

Beispiele:

\chcommand{usepackage[authormarkup=superscript]\{changes\}}\\
\chcommand{usepackage[authormarkup=subscript]\{changes\}}\\
\chcommand{usepackage[authormarkup=brackets]\{changes\}}\\
\chcommand{usepackage[authormarkup=footnote]\{changes\}}

\subsubsection{authormarkupposition}

Die \choption{authormarkupposition}-Option gibt an, wo die Autor-Identifizierung gesetzt wird.
Der default-Wert wird gewählt, wenn die Option nicht gesetzt wird.

Die folgenden Werte sind erlaubt:
\begin{description}
	\item [\choption{right}] rechts vom Text -- Text\textsuperscript{Beispiel} (default value)
	\item [\choption{left}] links vom Text -- \textsuperscript{Beispiel}Text
\end{description}

Beispiele:

\chcommand{usepackage[authormarkupposition=right]\{changes\}}\\
\chcommand{usepackage[authormarkupposition=left]\{changes\}}

\subsubsection{authormarkuptext}

Die \choption{authormarkuptext}-Option gibt an, was für die Autor-Identifizierung genutzt wird.
Der default-Wert wird gewählt, wenn die Option nicht gesetzt wird.

Die folgenden Werte sind erlaubt:
\begin{description}
	\item [\choption{id}] Autoren-ID -- Text\textsuperscript{ID} (default-Wert)
	\item [\choption{name}] Autorenname -- Text\textsuperscript{Autorenname}
\end{description}

Beispiele:

\chcommand{usepackage[authormarkuptext=id]\{changes\}}\\
\chcommand{usepackage[authormarkuptext=name]\{changes\}}

\subsubsection{ulem}

Optionen für das \chpackage{ulem}-Paket können als Parameter der \choption{ulem}-Option angegeben werden.
Zwei oder mehr Optionen müssen in geschweifte Klammern gesetzt werden.

\chcommand{usepackage[ulem=normalem]\{changes\}}\\
\chcommand{usepackage[ulem=\{normalem,normalbf\}]\{changes\}}

\subsubsection{xcolor}

Optionen für das \chpackage{xcolor}-Paket können als Parameter der \choption{xcolor}-Option angegeben werden.
Zwei oder mehr Optionen müssen in geschweifte Klammern gesetzt werden.

\chcommand{usepackage[xcolor=dvipdf]\{changes\}}\\
\chcommand{usepackage[xcolor=\{dvipdf,gray\}]\{changes\}}

%^^A ---- change management

\subsection{Änderungsmanagement}
\label{sec:user:changemanagement}

\subsubsection{\chcommand{added}}
\DescribeMacro{\added}

Der Befehl \chcommand{added} markiert zugefügten Text.
Der neue Text wird als notwendiges Argument in geschweiften Klammern übergeben.
Optional können eine Autoren-ID sowie eine Anmerkung übergeben werden.
Die Autoren-ID muss mit einer mit dem \chcommand{definechangesauthor}-Befehl definierten ID übereinstimmen.
Soll nur eine Anmerkung (ohne Autor) eingegeben werden, so ist statt des Autors ein leeres Argument zu übergeben.
\begin{einspiel}
\>\chcommand{added[\meta{Autor-ID}][\meta{Anmerkung}]\{\meta{neuer Text}\}}
\end{einspiel}
\begin{einspiel}[true]
\>\texttt{Das ist \chcommand{added}[EK]\{neuer\} Text.}\\
\>Das ist \added[EK]{neuer} Text.\\
\>\texttt{Das ist \chcommand{added}[][anonym]\{neuer\} Text.}\\
\>Das ist \added[][anonym]{neuer} Text.
\end{einspiel}

\subsubsection{\chcommand{deleted}}
\DescribeMacro{\deleted}

Der Befehl \chcommand{deleted} markiert gelöschten Text.
Argumente: siehe \chcommand{added}.
\begin{einspiel}
\>\chcommand{deleted[\meta{Autor-ID}][\meta{Anmerkung}]\{\meta{gelöschter Text}\}}
\end{einspiel}
\begin{einspiel}[true]
\>\texttt{Das ist \chcommand{deleted}[][obsolet]\{schlechter\} Text.}\\
\>Das ist \deleted[][obsolet]{schlechter} Text.
\end{einspiel}

\subsubsection{\chcommand{replaced}}
\DescribeMacro{\replaced}

Der Befehl \chcommand{replaced} markiert geänderten Text.
Notwendige Argumente sind der neue sowie der alte Text.
Optionale Argumente: siehe \chcommand{added}.
\begin{einspiel}
\>\chcommand{replaced[\meta{Autor-ID}][\meta{Anmerkung}]\{\meta{neuer Text}\}\{\meta{alter Text}\}}
\end{einspiel}
\begin{einspiel}[true]
\>\texttt{Das ist \chcommand{replaced}[EK]\{schöner\}\{schlechter\} Text.}\\
\>Das ist \replaced[EK]{schöner}{schlechter} Text.
\end{einspiel}

\subsubsection{\chcommand{listofchanges}}
\DescribeMacro{\listofchanges}

Der Befehl \chcommand{listofchanges} gibt eine Liste der Änderungen aus.
Im ersten \LaTeX-Lauf wird eine Hilfsdatei angelegt, deren Daten im zweiten Durchlauf eingebunden werden.
Für eine aktuelle Liste der Änderungen sind daher zwei \LaTeX-Läufe notwendig.
\begin{einspiel}
	\>\chcommand{listofchanges}
\end{einspiel}

%^^A ---- Author management

\subsection{Autorenverwaltung}
\label{sec:user:authormanagement}

\subsubsection{\chcommand{definechangesauthor}}
\DescribeMacro{\definechangesauthor}

Der Befehl \chcommand{definechangesauthor} definiert einen neuen Autor für Änderungen.
Es muss eine eindeutige Autor-ID angegeben werden, die keine Sonder- oder Leerzeichen enthalten darf.
Optional kann eine Farbe und ein Name angegeben werden.
Wird keine Farbe angegeben, wird schwarz genutzt.
Der Autorenname wird in der Änderungsliste benutzt und im Markup, wenn die entsprechende Option gesetzt ist.
\begin{einspiel}
	\>\chcommand{definechangesauthor[name=\{\meta{Autor-Name}\}, color=\{\meta{Farbe}\}]\{\meta{Autor-ID}\}}
\end{einspiel}

\begin{einspiel}[true]
	\>\chcommand{definechangesauthor\{EK\}}\\
	\>\chcommand{definechangesauthor[color=orange]\{EK\}}\\
	\>\chcommand{definechangesauthor[name=\{Ekkart Kleinod\}]\{EK\}}\\
	\>\chcommand{definechangesauthor[name=\{Ekkart Kleinod\}, color=orange]\{EK\}}
\end{einspiel}

%^^A ---- Adaptation of the output
\subsection{Anpassung der Ausgabe}
\label{sec:user:customizingoutput}

\subsubsection{\chcommand{setauthormarkup}}
\DescribeMacro{\setauthormarkup}

Der Befehl \chcommand{setauthormarkup} legt fest, wie der Autor im Text angezeigt wird.
Ohne andere Definition gilt, dass der Autor rechts von den Änderungen hochgestellt erscheint.

Werte für Position (optional): \emph{left} == links von den Änderungen; alles andere: rechts\\
Werte für Definition: beliebige \LaTeX-Befehle, der Autorenname wird mit "`\#1"' gekennzeichnet.
\begin{einspiel}
	\>\chcommand{setauthormarkup[\meta{position}]\{\meta{definition}\}}
\end{einspiel}
\begin{einspiel}[true]
	\>\chcommand{setauthormarkup\{(\#1)\}}\\
	\>\chcommand{setauthormarkup[left]\{(\#1)\textasciitilde{}-{}-\textasciitilde{}\}}\\
	\>\chcommand{setauthormarkup\{}\chcommand{marginpar\{\#1\}\}}\\
	\>\chcommand{setauthormarkup[right]\{\}}
\end{einspiel}

\subsubsection{\chcommand{setremarkmarkup}}
\DescribeMacro{\setremarkmarkup}

Der Befehl \chcommand{setremarkmarkup} legt fest, wie die Anmerkungen im Text angezeigt werden.
Ohne andere Definition gilt, dass die Anmerkungen als Fußnote mit farbigem Text gesetzt werden.

Werte für Definition: beliebige \LaTeX-Befehle, die Autor-ID wird mit "`\#1"' benutzt, der Anmerkungstext mit "`\#2"'.
Über die Autor-ID kann mit \texttt{Changes@Color\#1} die Farbe des Autors benutzt werden.
\begin{einspiel}
	\>\chcommand{setremarkmarkup\{\meta{definition}\}}
\end{einspiel}
\begin{einspiel}[true]
	\>\chcommand{setremarkmarkup\{(\#2:\#1)\}}\\
	\>\chcommand{setremarkmarkup\{\chcommand{footnote}\{\#1:\chcommand{textcolor\{Changes@Color\#1\}}\{\#2\}\}\}}
\end{einspiel}

\subsubsection{\chcommand{setlocextension}}
\DescribeMacro{\setlocextension}

Der Befehl \chcommand{setlocextension} legt das Suffix der Hilfsdatei für die Änderungsliste (loc-Datei\footnote{%
	"`loc"' steht dabei für "`list of changes"'.
}) fest.
Ohne andere Definition gilt das Suffix "`\texttt{loc}"'.
Das Beispiel würde für "`\texttt{foo.tex}"' Hilfsdateien erzeugen, die "`\texttt{foo.changes}"' statt des Standardnamens "`\texttt{foo.loc}"' heißen.
\begin{einspiel}
	\>\chcommand{setlocextension\{\meta{extension}\}}
\end{einspiel}
\begin{einspiel}[true]
	\>\chcommand{setlocextension\{changes\}}
\end{einspiel}

%^^A ---- packages
\subsection{Benötigte Pakete}
\label{sec:user:packages}

Das \chpackage{changes}-Paket bindet bereits Pakete ein, die für das Paket notwendig sind.
Eine genauere Beschreibung der einzelnen Pakete ist in der Dokumentation der Pakete selbst zu finden.

Die folgenden Pakete sind zwingend notwendig müssen für die Nutzung des \chpackage{changes}-Pakets installiert sein:
\begin{description}
	\item [ifthen] stellt eine verbesserte \texttt{if}-Abfrage sowie eine \texttt{while}-Schleife zur Verfügung
	\item [xkeyval] Eingabe von Optionen mit Werteübergabe
\end{description}

Die folgenden Pakete sind manchmal notwendig und müssen installiert sein, wenn sie genutzt werden:
\begin{description}
	\item [pdfcolmk] wird geladen, wenn farbiger Text genutzt wird (default Markup); löst das Problem farbigen Texts über Seitenumbrüche hinweg (bei pdflatex)
	\item [ulem] wird geladen, wenn Text durchgestrichen oder ausge-x-t wird (default Markup); Durchstreichen von Texten
	\item [xcolor] wird geladen, wenn farbiger Text genutzt wird (default Markup); farbige Markierung von Texten
\end{description}


%^^A ---- Authors
\section{Autoren}
\label{sec:authors}

Am \chpackage{changes}-Paket haben mehrere Autoren mitgewirkt.
Dies sind in alphabetischer Reihenfolge:
\begin{itemize}
	\item Chiaradonna, Silvano
	\item Giovannini, Daniele
	\item Kleinod, Ekkart
	\item Wölfel, Philipp
	\item Wolter, Steve
\end{itemize}



%^^A ---- Versions
\section{Versionen}
\label{sec:versions}

\minisec{Version 0.6.0}

Datum: ??.\,0?.~2011
\begin{itemize}
	\item Italienische Übersetzungen der captions von Daniele Giovannini
	\item neues Nutzerinterface für das Setzen von Optionen sowie die Definition von Markup und Autoren
\end{itemize}

\minisec{Version 0.5.4}

Datum: 25.\,04.~2011
\begin{itemize}
	\item Auslagerung der Nutzerdokumentation in eigene Datei
	\item Änderung der default-Sprache zu Englisch
	\item neues Script, um die \chpackage{changes}-Befehle zu löschen von Silvano Chiaradonna
\end{itemize}

\minisec{Version 0.5.3}

Datum: 22.\,11.~2010
\begin{itemize}
\item Dokumentoptionen von \chcommand{documentclass} werden ebenfalls genutzt (Vorschlag und Code von Steve Wolter)
\end{itemize}

\minisec{Version 0.5.2}

Datum: 10.\,10.~2007
\begin{itemize}
	\item Paketoptionen der Pakete \chpackage{pdfcolmk}, \chpackage{ulem}, and \chpackage{xcolor} werden weitergeleitet
\end{itemize}

\minisec{Version 0.5.1}

Datum: 27.\,08.~2007
\begin{itemize}
	\item gelöschter Text wieder durchgestrichen, Paket \chpackage{ulem} funktioniert; ausgrauen hat nicht funktioniert
\end{itemize}

\minisec{Version 0.5}

Datum: 26.\,08.~2007
\begin{itemize}
	\item keine Nutzung des \chpackage{arrayjob}-Pakets mehr, dadurch Fehler im Zusammenspiel mit \chpackage{array} behoben
	\item auf UTF-8-encoding umgestellt
	\item keine Nutzung des \chpackage{soul}-Pakets mehr, dadurch Fehler im Zusammenspiel UTF-8-encoding behoben
	\item gelöschter Text durch grauen Hintergrund visualisiert (es gibt bisher kein ordentliches Durchstreichen bei UTF-8-Nutzung)
	\item neues optionales Argument für Autorenname
	\item farbige Liste der Änderungen
	\item loc-Format geändert
	\item englische Doku verbessert
\end{itemize}

\minisec{Version 0.4}

Datum: 24.\,01.~2007
\begin{itemize}
	\item \chpackage{pdfcolmk} eingebunden, um Problem mit farbigem Text bei Seitenumbrüchen zu lösen
	\item \chcommand{setremarkmarkup} um Autor-ID erweitert, um Anmerkung farbig setzen zu können
	\item Anmerkungen werden in der Fußnote farbig gesetzt
	\item erste Version für das CTAN
\end{itemize}

\minisec{Version 0.3}

Datum: 22.\,01.~2007
\begin{itemize}
	\item englische Nutzerdokumentation
	\item Befehl \chcommand{changed} ersetzt durch \chcommand{replaced}
	\item verbesserte \choption{final}-Option: kein zusätzlicher Leerraum
\end{itemize}

\minisec{Version 0.2}

Datum: 17.\,01.~2007
\begin{itemize}
	\item Bezeichnungen auch bei fehlendem \chpackage{babel}-Paket eingeführt
	\item \chcommand{setauthormarkup}, \chcommand{setlocextension}, \chcommand{setremarkmarkup} für Einstellungen
	\item Beispieldateien generiert
	\item LPPL eingefügt
\end{itemize}
Beseitigte Fehler
\begin{itemize}
	\item Fehler mit \chpackage{ifthen}-Paketplazierung behoben
	\item bei Liste war immer "`Eingefügt"' eingestellt, behoben
	\item Autorausgabe war buggy (\chcommand{if}-Abfrage nicht einwandfrei)
\end{itemize}

\minisec{Version 0.1}

Datum: 16.\,01.~2007
\begin{itemize}
	\item initiale Version
	\item Befehle \chcommand{added}, \chcommand{deleted} und \chcommand{changed}
\end{itemize}

\section{Weitergabe, Copyright, Lizenz}

Copyright 2007-2011 Ekkart Kleinod (\href{mailto:ekleinod@edgesoft.de}{ekleinod@edgesoft.de})

Dieses Paket darf unter der "`\LaTeX\ Project Public License"' Version~1.3 oder jeder späteren Version weitergegeben und/oder geändert werden.
Die neueste Version dieser Lizenz steht auf\\
\url{http://www.latex-project.org/lppl.txt}\\
Version~1.3 und spätere Versionen sind Teil aller \LaTeX-Distributionen ab Version~2005/12/01.

Dieses Paket besitzt den Status "`maintained"' (verwaltet).
Der aktuelle Verwalter dieses Pakets ist Ekkart Kleinod.

Dieses Paket besteht aus den Dateien

\begin{tabbing}
	mm\=\kill
	\>\texttt{source/latex/changes/changes.drv}\\
	\>\texttt{source/latex/changes/changes.dtx}\\
	\>\texttt{source/latex/changes/changes.ins}\\
	\>\texttt{source/latex/changes/README}\\
	\>\texttt{source/latex/changes/examples.dtx}\\

	\>\texttt{scripts/changes/delcmdchanges.bash}
\end{tabbing}

und den generierten Dateien

\begin{tabbing}
	mm\=\kill
	\>\texttt{doc/latex/changes/changes.english.full.pdf}\\
	\>\texttt{doc/latex/changes/changes.english.short.pdf}\\
	\>\texttt{doc/latex/changes/changes.ngerman.full.pdf}\\
	\>\texttt{doc/latex/changes/changes.ngerman.short.pdf}\\

	\>\texttt{doc/latex/changes/examples/changes.example.*.tex}\\
	\>\texttt{doc/latex/changes/examples/changes.example.*.pdf}\\

	\>\texttt{tex/latex/changes/changes.sty}
\end{tabbing}

%^^A end of user documentation


% \fi
%
%^^A -- source code
%
% \StopEventually
%
% \section{The documented sourcecode}
%
% \iflanguage{english}{}{
%  The sourcecode is documented in English only.
%  This is intended, please do not provide translations for the text below, just corrections or improvements.
% }
%
%    \begin{macrocode}
%<*changes>
%    \end{macrocode}
%
% \subsection{Package information and options}
%
% Set needed \LaTeX-format to \LaTeXe{}, provide name, date, version.
% Type some information to the console.
%    \begin{macrocode}
\NeedsTeXFormat{LaTeX2e}
\ProvidesPackage{changes}
[2011/10/19 v0.6.0 changes-Paket]
\typeout{*** changes-Paket 2011/10/19 v0.6.0 ***}
%    \end{macrocode}
%
% Package \chpackage{xkeyval} provides options with key-value-pairs.
%    \begin{macrocode}
\RequirePackage{xkeyval}
%    \end{macrocode}
%
% Package \chpackage{xifthen} provides improved \texttt{if} as well as a \texttt{while}-loop.
%    \begin{macrocode}
\RequirePackage{xifthen}
%    \end{macrocode}
%
% \subsubsection{Package options}
%
% Option \choption{draft}, \emph{default} is \texttt{true}.
%    \begin{macrocode}
\newboolean{Changes@optiondraft}
\setboolean{Changes@optiondraft}{true}
\DeclareOptionX{draft}{
	\setboolean{Changes@optiondraft}{true}
	\typeout{changes-option '\CurrentOption'}
}
%    \end{macrocode}
%
% Option \choption{final}, sets \choption{draft} to \texttt{false}.
%    \begin{macrocode}
\DeclareOptionX{final}{
	\setboolean{Changes@optiondraft}{false}
	\typeout{changes-option '\CurrentOption'}
}
%    \end{macrocode}
%
% Declare storage for markup options, they are set by the markup option but can be changed with the mor special options, therefore they have to be declared at this place.
%    \begin{macrocode}
\newcommand{\Changes@optionaddedmarkup}{none}
\newcommand{\Changes@optiondeletedmarkup}{sout}
%    \end{macrocode}
%
% Option \choption{markup}, sets markup options accordingly.
%    \begin{macrocode}
\newcommand{\Changes@optionmarkup}{default}
\DeclareOptionX{markup}{
	\ifthenelse{\equal{\@empty}{#1}}
		{}
		{
			\ifthenelse{
				\equal{#1}{default}\or
				\equal{#1}{underlined}\or
				\equal{#1}{bfit}\or
				\equal{#1}{nocolor}
			}
				{\renewcommand{\Changes@optionmarkup}{#1}}
				{\PackageWarning{changes}{markup '#1' unknown, using '\Changes@optionmarkup'}}
		}
	\ifthenelse{\equal{\Changes@optionmarkup}{default}}
		{
			\renewcommand{\Changes@optionaddedmarkup}{none}
			\renewcommand{\Changes@optiondeletedmarkup}{sout}
		}
		{}
	\ifthenelse{\equal{\Changes@optionmarkup}{underlined}}
		{
			\renewcommand{\Changes@optionaddedmarkup}{uline}
			\renewcommand{\Changes@optiondeletedmarkup}{sout}
		}
		{}
	\ifthenelse{\equal{\Changes@optionmarkup}{bfit}}
		{
			\renewcommand{\Changes@optionaddedmarkup}{bf}
			\renewcommand{\Changes@optiondeletedmarkup}{it}
		}
		{}
	\ifthenelse{\equal{\Changes@optionmarkup}{nocolor}}
		{
			\renewcommand{\Changes@optionaddedmarkup}{uline}
			\renewcommand{\Changes@optiondeletedmarkup}{sout}
		}
		{}
	\typeout{changes-option 'markup=\Changes@optionmarkup'}
}
%    \end{macrocode}
%
% Option \choption{addedmarkup}, stored or set to default value \texttt{none}.
%    \begin{macrocode}
\DeclareOptionX{addedmarkup}{
	\ifthenelse{\equal{\@empty}{#1}}
		{}
		{
			\ifthenelse{
				\equal{#1}{none}\or
				\equal{#1}{uline}\or
				\equal{#1}{uuline}\or
				\equal{#1}{uwave}\or
				\equal{#1}{dashuline}\or
				\equal{#1}{dotuline}\or
				\equal{#1}{sout}\or
				\equal{#1}{xout}\or
				\equal{#1}{bf}\or
				\equal{#1}{it}\or
				\equal{#1}{sl}\or
				\equal{#1}{em}
			}
				{\renewcommand{\Changes@optionaddedmarkup}{#1}}
				{\PackageWarning{changes}{addedmarkup '#1' unknown, using '\Changes@optionaddedmarkup'}}
		}
	\typeout{changes-option 'addedmarkup=\Changes@optionaddedmarkup'}
}
%    \end{macrocode}
%
% Option \choption{deletedmarkup}, stored or set to default value \texttt{striked}.
%    \begin{macrocode}
\DeclareOptionX{deletedmarkup}{
	\ifthenelse{\equal{\@empty}{#1}}
		{}
		{
			\ifthenelse{
				\equal{#1}{none}\or
				\equal{#1}{uline}\or
				\equal{#1}{uuline}\or
				\equal{#1}{uwave}\or
				\equal{#1}{dashuline}\or
				\equal{#1}{dotuline}\or
				\equal{#1}{sout}\or
				\equal{#1}{xout}\or
				\equal{#1}{bf}\or
				\equal{#1}{it}\or
				\equal{#1}{sl}\or
				\equal{#1}{em}
			}
				{\renewcommand{\Changes@optiondeletedmarkup}{#1}}
				{\PackageWarning{changes}{deletedmarkup '#1' unknown, using '\Changes@optiondeletedmarkup'}}
		}
	\typeout{changes-option 'deletedmarkup=\Changes@optiondeletedmarkup'}
}
%    \end{macrocode}
%
% Declare storage for authormarkup option and store option value or set to default value \texttt{superscript}.
%    \begin{macrocode}
\newcommand{\Changes@optionauthormarkup}{superscript}
\DeclareOptionX{authormarkup}{
	\ifthenelse{\equal{\@empty}{#1}}
		{}
		{
			\ifthenelse{
				\equal{#1}{superscript}\or
				\equal{#1}{subscript}\or
				\equal{#1}{brackets}\or
				\equal{#1}{footnote}
			}
				{\renewcommand{\Changes@optionauthormarkup}{#1}}
				{\PackageWarning{changes}{authormarkup '#1' unknown, using '\Changes@optionauthormarkup'}}
		}
	\typeout{changes-option 'authormarkup=\Changes@optionauthormarkup'}
}
%    \end{macrocode}
%
% Declare storage for authormarkupposition option and store option value or set to default value \texttt{right}.
%    \begin{macrocode}
\newcommand{\Changes@optionauthormarkupposition}{right}
\DeclareOptionX{authormarkupposition}{
	\ifthenelse{\equal{\@empty}{#1}}
		{}
		{
			\ifthenelse{
				\equal{#1}{right}\or
				\equal{#1}{left}
			}
				{\renewcommand{\Changes@optionauthormarkupposition}{#1}}
				{\PackageWarning{changes}{authormarkupposition '#1' unknown, using '\Changes@optionauthormarkupposition'}}
		}
	\typeout{changes-option 'authormarkupposition=\Changes@optionauthormarkupposition'}
}
%    \end{macrocode}
%
% Declare storage for authormarkuptext option and store option value or set to default value \texttt{id}.
%    \begin{macrocode}
\newcommand{\Changes@optionauthormarkuptext}{id}
\DeclareOptionX{authormarkuptext}{
	\ifthenelse{\equal{\@empty}{#1}}
		{}
		{
			\ifthenelse{
				\equal{#1}{id}\or
				\equal{#1}{name}
			}
				{\renewcommand{\Changes@optionauthormarkuptext}{#1}}
				{\PackageWarning{changes}{authormarkuptext '#1' unknown, using '\Changes@optionauthormarkuptext'}}
		}
	\typeout{changes-option 'authormarkuptext=\Changes@optionauthormarkuptext'}
}
%    \end{macrocode}
%
%
%
% Options for package \chpackage{ulem} are directly passed to the package.
%    \begin{macrocode}
\DeclareOptionX{ulem}{
	\typeout{ulem-option '#1', passed to package ulem}
	\PassOptionsToPackage{#1}{ulem}
}
%    \end{macrocode}
%
% Options for package \chpackage{xcolor} are directly passed to the package.
%    \begin{macrocode}
\DeclareOptionX{xcolor}{
	\typeout{xcolor-option '#1', passed to package xcolor}
	\PassOptionsToPackage{#1}{xcolor}
}
%    \end{macrocode}
%
% Unknown options generate a package warning.
%    \begin{macrocode}
\DeclareOptionX*{
	\PackageWarning{changes}{Unknown option '\CurrentOption'}
}
%    \end{macrocode}
%
% \subsubsection{Command options}
%
% All options for commands (e.g. \chcommand{definechangesauthor}) have to be declared before option processing.
%
% \minisec{\chcommand{definechangesauthor}}
%
% Declare available options of the command, define value containers.
%    \begin{macrocode}
\DeclareOptionX<Changes@definechangesauthor>{name}{\def\Changes@definechangesauthor@name{#1}}
\DeclareOptionX<Changes@definechangesauthor>{color}{\def\Changes@definechangesauthor@color{#1}}
%    \end{macrocode}
%
% Set the default values of the options.
%    \begin{macrocode}
\ExecuteOptionsX<Changes@definechangesauthor>{%
	name=\@empty,%
	color=black
}
%    \end{macrocode}
%
% \subsubsection{Option processing}
%
% Process the options.
%    \begin{macrocode}
\ProcessOptionsX*\relax
%    \end{macrocode}
%
% \subsection{Packages}
%
% Package \chpackage{xcolor} provides colored text.
% Package \chpackage{pdfcolmk} solves the problem of colored text and page breaks (has to be loaded after \chpackage{xcolor}).
%    \begin{macrocode}
\newboolean{Changes@colored}
\setboolean{Changes@colored}{true}
\ifthenelse{\equal{\Changes@optionmarkup}{nocolor}}
	{\setboolean{Changes@colored}{false}}
	{}
\ifthenelse{\boolean{Changes@colored}}
	{
		\RequirePackage{xcolor}
		\RequirePackage{pdfcolmk}
	}
	{}
%    \end{macrocode}
%
% Package \chpackage{ulem} provides commands for striking out text.
%    \begin{macrocode}
\ifthenelse{
	\equal{\Changes@optionaddedmarkup}{uline}\or
	\equal{\Changes@optionaddedmarkup}{uuline}\or
	\equal{\Changes@optionaddedmarkup}{uwave}\or
	\equal{\Changes@optionaddedmarkup}{dashuline}\or
	\equal{\Changes@optionaddedmarkup}{dotuline}\or
	\equal{\Changes@optionaddedmarkup}{sout}\or
	\equal{\Changes@optionaddedmarkup}{xout}\or
	\equal{\Changes@optiondeletedmarkup}{uline}\or
	\equal{\Changes@optiondeletedmarkup}{uuline}\or
	\equal{\Changes@optiondeletedmarkup}{uwave}\or
	\equal{\Changes@optiondeletedmarkup}{dashuline}\or
	\equal{\Changes@optiondeletedmarkup}{dotuline}\or
	\equal{\Changes@optiondeletedmarkup}{sout}\or
	\equal{\Changes@optiondeletedmarkup}{xout}
}
	{\RequirePackage[normalem,normalbf]{ulem}}
	{}
%    \end{macrocode}
%
% \subsection{Language dependent texts}
%
% Declaration of language dependent names and identifiers.
% The check for \chcommand{addto} is a check for the \chpackage{babel} package.
% If the babel package is not loaded, the default language is English, in order to use an own language, the user has to redefine the variables.
%    \begin{macrocode}
\ifthenelse{\isundefined{\addto}}
	{
		\def\listchangesname{Changes}
		\def\changesaddname{Added}
		\def\changesdeletename{Deleted}
		\def\changesreplacename{Replaced}
		\def\changesauthorname{Author}
		\def\changesanonymousname{anonymous}
		\def\changesnoloc{List of changes is available after the next \LaTeX\ run.}
	}{
		\addto\captionsngerman{\def\listchangesname{\"Anderungen}}
		\addto\captionsngerman{\def\changesaddname{Eingef\"ugt}}
		\addto\captionsngerman{\def\changesdeletename{Gel\"oscht}}
		\addto\captionsngerman{\def\changesreplacename{Ersetzt}}
		\addto\captionsngerman{\def\changesauthorname{Autor}}
		\addto\captionsngerman{\def\changesanonymousname{Anonym}}
		\addto\captionsngerman{\def\changesnoloc{\"Anderungsliste nach dem n\"achsten \LaTeX-Lauf verf\"ugbar.}}

		\addto\captionsgerman{\def\listchangesname{\"Anderungen}}
		\addto\captionsgerman{\def\changesaddname{Eingef\"ugt}}
		\addto\captionsgerman{\def\changesdeletename{Gel\"oscht}}
		\addto\captionsgerman{\def\changesreplacename{Ersetzt}}
		\addto\captionsgerman{\def\changesauthorname{Autor}}
		\addto\captionsgerman{\def\changesanonymousname{Anonym}}
		\addto\captionsgerman{\def\changesnoloc{\"Anderungsliste nach dem n\"achsten \LaTeX-Lauf verf\"ugbar.}}

		\addto\captionsenglish{\def\listchangesname{Changes}}
		\addto\captionsenglish{\def\changesaddname{Added}}
		\addto\captionsenglish{\def\changesdeletename{Deleted}}
		\addto\captionsenglish{\def\changesreplacename{Replaced}}
		\addto\captionsenglish{\def\changesauthorname{Author}}
		\addto\captionsenglish{\def\changesanonymousname{anonymous}}
		\addto\captionsenglish{\def\changesnoloc{List of changes is available after the next \LaTeX\ run.}}

		\addto\captionsitalian{\def\listchangesname{Modifiche}}
		\addto\captionsitalian{\def\changesaddname{Aggiunte}}
		\addto\captionsitalian{\def\changesdeletename{Cancellazioni}}
		\addto\captionsitalian{\def\changesreplacename{Sostituzioni}}
		\addto\captionsitalian{\def\changesauthorname{Autore}}
		\addto\captionsitalian{\def\changesanonymousname{anonimo}}
		\addto\captionsitalian{\def\changesnoloc{La lista delle modifiche sar\`a disponibile alla prossima esecuzione di \LaTeX.}}
	}
%    \end{macrocode}
%
% \subsection{File extension}
%
% \begin{macro}{\Changes@extension}
% Store file extension in variable, set default to \texttt{loc}.
%    \begin{macrocode}
\newcommand{\Changes@extension}{loc}
%    \end{macrocode}
% \end{macro}
%
% \begin{macro}{\setlocextension}
%  Sets a new file extension.
%  Argument: new extension.
%    \begin{macrocode}
\newcommand{\setlocextension}[1]{
	\renewcommand{\Changes@extension}{#1}
}
%    \end{macrocode}
% \end{macro}
%
%
% \subsection{Authors}
%
% \subsubsection{Author management}
%
% Author counter.
%    \begin{macrocode}
\newcounter{Changes@AuthorCount}
\setcounter{Changes@AuthorCount}{0}
\newcounter{Changes@iAuthor}
%    \end{macrocode}
%
% \begin{macro}{\definechangesauthor}
%  Define a new author.
%  Mandatory argument: author's id.
%  Optional arguments (key-value): author's name (default: empty) and author's color (default: black).
%
%  Store id, name and color using named variables.
%  Define counter and color per author.
%    \begin{macrocode}
\newcommand*\definechangesauthor[2][]{%
%    \end{macrocode}
%
% Call \emph{setkeys} in order to evaluate the key-value-options and fill the value containers.
%    \begin{macrocode}
	\setkeys{Changes@definechangesauthor}{#1}
%    \end{macrocode}
%
% Increment author counter, later needed for \emph{while} loop of authors.
%    \begin{macrocode}
	\stepcounter{Changes@AuthorCount}
%    \end{macrocode}
%
% Store the id in a name with the given counter/index.
% All other storages refer to the id.
%    \begin{macrocode}
	\@namedef{Changes@AuthorID\theChanges@AuthorCount}{#2}
%    \end{macrocode}
%
% Store the author's definition in according variables/colors, create change counters.
%    \begin{macrocode}
	%\@namedef{Changes@AuthorName#2}{\Changes@definechangesauthor@name}
	\expandafter
	\let\csname Changes@AuthorName#2\endcsname=\Changes@definechangesauthor@name
	\newcounter{Changes@AddCount#2}
	\newcounter{Changes@DeleteCount#2}
	\newcounter{Changes@ReplaceCount#2}
	\ifthenelse{\boolean{Changes@colored}}
		{
			%\@namedef{Changes@AuthorColor#2}{\Changes@definechangesauthor@color}
			\expandafter
			\let\csname Changes@AuthorColor#2\endcsname=\Changes@definechangesauthor@color
			\colorlet{Changes@Color#2}{\@nameuse{Changes@AuthorColor#2}}
		}
		{}
}
%    \end{macrocode}
% \end{macro}
%
% Define default-author (anonymous) with empty id and blue color.
%    \begin{macrocode}
\definechangesauthor[color=blue]{\@empty}
%    \end{macrocode}
%
%
% \subsubsection{Author markup}
%
% \begin{macro}{\Changes@Markup@Author}
% Store markup for authors.
%    \begin{macrocode}
\newcommand{\Changes@Markup@Author}[1]{%
	\ifthenelse{\equal{\Changes@optionauthormarkup}{superscript}}{\textsuperscript{#1}}{}%
	\ifthenelse{\equal{\Changes@optionauthormarkup}{subscript}}{\textsubscript{#1}}{}%
	\ifthenelse{\equal{\Changes@optionauthormarkup}{brackets}}{(#1)}{}%
	\ifthenelse{\equal{\Changes@optionauthormarkup}{footnote}}{\footnote{#1}}{}%
}
%    \end{macrocode}
% \end{macro}
%
% \begin{macro}{\setauthormarkup}
% Set markup for authors.
%    \begin{macrocode}
\newcommand{\setauthormarkup}[1]{
	\renewcommand{\Changes@Markup@Author}[1]{#1}
}
%    \end{macrocode}
% \end{macro}
%
% \begin{macro}{\textsubscript}
% Define the command \chcommand{textsubscript} in case the author markup \choption{subscript} is used and the command \chcommand{textsubscript} is not defined yet.
% \chcommand{textsubscript} is the antagonist of \chcommand{textsuperscript} which is predefined in \LaTeX.
% The code is taken from \LaTeX-FAQ 8.5.17.
%    \begin{macrocode}
\ifthenelse{\isundefined{\textsubscript}}
 {
	\DeclareRobustCommand*\textsubscript[1]{\@textsubscript{\selectfont#1}}
	\newcommand{\@textsubscript}[1]{{\m@th\ensuremath{_{\mbox{\fontsize\sf@size\z@#1}}}}}
 }{}
%    \end{macrocode}
% \end{macro}
%
% \begin{macro}{\setauthormarkupposition}
% Set position for author markup text.
%    \begin{macrocode}
\newcommand{\setauthormarkupposition}[1]{
	\renewcommand{\Changes@optionauthormarkupposition}{#1}
}
%    \end{macrocode}
% \end{macro}
%
% \begin{macro}{\setauthormarkuptext}
% Set author markup text to be displayed.
%    \begin{macrocode}
\newcommand{\setauthormarkuptext}[1]{
	\renewcommand{\Changes@optionauthormarkuptext}{#1}
}
%    \end{macrocode}
% \end{macro}
%
% \begin{macro}{\Changes@Remark}
%  Markup of remarks.
%  Default: in a footnote.
%    \begin{macrocode}
\newcommand{\Changes@Remark}[2]{%
	\footnote{#1: #2}%
}
%    \end{macrocode}
% \end{macro}
%
% \begin{macro}{\setremarkmarkup}
%  Redefining the remark markup.
%  Mandatory argument: markup definition.
%    \begin{macrocode}
\newcommand{\setremarkmarkup}[1]{%
	\renewcommand{\Changes@Remark}[2]{#1}%
}
%    \end{macrocode}
% \end{macro}
%
% \subsection{Change management commands}
%
% \subsubsection{Text markup definition}
%
% \begin{macro}{\Changes@Markup@Added}
% Store markup for added text.
%    \begin{macrocode}
\newcommand{\Changes@Markup@Added}[1]{%
	\ifthenelse{\equal{\Changes@optionaddedmarkup}{none}}{#1}{}%
	\ifthenelse{\equal{\Changes@optionaddedmarkup}{uline}}{\uline{#1}}{}%
	\ifthenelse{\equal{\Changes@optionaddedmarkup}{uuline}}{\uuline{#1}}{}%
	\ifthenelse{\equal{\Changes@optionaddedmarkup}{uwave}}{\uwave{#1}}{}%
	\ifthenelse{\equal{\Changes@optionaddedmarkup}{dashuline}}{\dashuline{#1}}{}%
	\ifthenelse{\equal{\Changes@optionaddedmarkup}{dotuline}}{\dotuline{#1}}{}%
	\ifthenelse{\equal{\Changes@optionaddedmarkup}{sout}}{\sout{#1}}{}%
	\ifthenelse{\equal{\Changes@optionaddedmarkup}{xout}}{\xout{#1}}{}%
	\ifthenelse{\equal{\Changes@optionaddedmarkup}{bf}}{\textbf{#1}}{}%
	\ifthenelse{\equal{\Changes@optionaddedmarkup}{it}}{\textit{#1}}{}%
	\ifthenelse{\equal{\Changes@optionaddedmarkup}{sl}}{\textsl{#1}}{}%
	\ifthenelse{\equal{\Changes@optionaddedmarkup}{em}}{\emph{#1}}{}%
}
%    \end{macrocode}
% \end{macro}
%
% \begin{macro}{\setaddedmarkup}
% Set markup for added text.
%    \begin{macrocode}
\newcommand{\setaddedmarkup}[1]{
	\renewcommand{\Changes@Markup@Added}[1]{#1}
}
%    \end{macrocode}
% \end{macro}
%
% \begin{macro}{\Changes@Markup@Deleted}
% Store markup for deleted text.
%    \begin{macrocode}
\newcommand{\Changes@Markup@Deleted}[1]{%
	\ifthenelse{\equal{\Changes@optiondeletedmarkup}{none}}{#1}{}%
	\ifthenelse{\equal{\Changes@optiondeletedmarkup}{uline}}{\uline{#1}}{}%
	\ifthenelse{\equal{\Changes@optiondeletedmarkup}{uuline}}{\uuline{#1}}{}%
	\ifthenelse{\equal{\Changes@optiondeletedmarkup}{uwave}}{\uwave{#1}}{}%
	\ifthenelse{\equal{\Changes@optiondeletedmarkup}{dashuline}}{\dashuline{#1}}{}%
	\ifthenelse{\equal{\Changes@optiondeletedmarkup}{dotuline}}{\dotuline{#1}}{}%
	\ifthenelse{\equal{\Changes@optiondeletedmarkup}{sout}}{\sout{#1}}{}%
	\ifthenelse{\equal{\Changes@optiondeletedmarkup}{xout}}{\xout{#1}}{}%
	\ifthenelse{\equal{\Changes@optiondeletedmarkup}{bf}}{\textbf{#1}}{}%
	\ifthenelse{\equal{\Changes@optiondeletedmarkup}{it}}{\textit{#1}}{}%
	\ifthenelse{\equal{\Changes@optiondeletedmarkup}{sl}}{\textsl{#1}}{}%
	\ifthenelse{\equal{\Changes@optiondeletedmarkup}{em}}{\emph{#1}}{}%
}
%    \end{macrocode}
% \end{macro}
%
% \begin{macro}{\setdeletedmarkup}
% Set markup for added text.
%    \begin{macrocode}
\newcommand{\setdeletedmarkup}[1]{
	\renewcommand{\Changes@Markup@Deleted}[1]{#1}
}
%    \end{macrocode}
% \end{macro}
%
%
% \subsubsection{Change management command definition}
%
% \begin{macro}{\Changes@output}
% Output command for the changed text.
% This command has the following arguments:
% \begin{enumerate}
%		\item changed text (including markup)
%		\item unchanged text
%		\item author's id
%		\item remark
% \end{enumerate}
%    \begin{macrocode}
\newcommand{\Changes@output}[4]{%
%    \end{macrocode}
%	Output changed text if option \choption{draft} is set, otherwise output unchanged text.
%    \begin{macrocode}
	\ifthenelse{\boolean{Changes@optiondraft}}%
		{%
%    \end{macrocode}
%	Save author text for output.
%    \begin{macrocode}
			\ifthenelse{\equal{\Changes@optionauthormarkuptext}{id}}%
				{\@namedef{Changes@AuthorText}{#3}}%
				{}%
			\ifthenelse{\equal{\Changes@optionauthormarkuptext}{name}}%
				{\@namedef{Changes@AuthorText}{\@nameuse{Changes@AuthorName#3}}}%
				{}%
			{%
%    \end{macrocode}
%	Change color, if colored text is used.
%    \begin{macrocode}
				\ifthenelse{\boolean{Changes@colored}}%
					{\color{Changes@Color#3}}%
					{}%
%    \end{macrocode}
%	Output author text if author's id is given and text should appear left of changes.
%    \begin{macrocode}
				\ifthenelse{\equal{\Changes@optionauthormarkupposition}{left} \and \not\equal{#3}{\@empty}}%
					{\Changes@Markup@Author{\@nameuse{Changes@AuthorText}}}%
					{}%
%    \end{macrocode}
%	Output changed text.
%    \begin{macrocode}
				{#1}%
%    \end{macrocode}
%	Output author text if author's id is given and text should appear right of changes.
%    \begin{macrocode}
				\ifthenelse{\equal{\Changes@optionauthormarkupposition}{right} \and \not\equal{#3}{\@empty}}%
					{\Changes@Markup@Author{\@nameuse{Changes@AuthorText}}}%
					{}%
%    \end{macrocode}
%	Output remark if a remark is given.
%    \begin{macrocode}
				\ifthenelse{\not\equal{#4}{\@empty} \and \not\equal{#3}{\@empty}}%
					{\Changes@Remark{#3}{#4}}%
					{}%
			}%
		}%
%    \end{macrocode}
%	Output unchanged text (option \choption{final} was set).
%    \begin{macrocode}
		{#2}%
}
%    \end{macrocode}
% \end{macro}
%
% \begin{macro}{\Changes@temp}
% Temporary variable for exchange of optional parameters between interlocked commands.
%    \begin{macrocode}
\newcommand{\Changes@temp}{\@empty}
%    \end{macrocode}
% \end{macro}
%
% \begin{macro}{\added}
%  The command formats text as new text.
%  It's rather complicated for defining a command with two optional parameters.
%
%  Mandatory argument: added text.
%  Optional arguments: author's id, remark
%    \begin{macrocode}
\newcommand{\added}[1][\@empty]{%
	\renewcommand{\Changes@temp}{#1}%
	\Changes@added%
}
\newcommand{\Changes@added}[2][\@empty]{%
	\Changes@output
		{\Changes@Markup@Added{#2}}
		{#2}
		{\Changes@temp}
		{#1}%
	\stepcounter{Changes@AddCount\Changes@temp}%
}
%    \end{macrocode}
% \end{macro}
%
% \begin{macro}{\deleted}
%  The command formats text as deleted text.
%  It's rather complicated for defining a command with two optional parameters.
%
%  The \choption{final}-part is taken from a tip from \texttt{de.comp.text.tex}.
%  It solves the problem of additional space caused by an empty command.
%
%  Mandatory argument: deleted text.
%  Optional arguments: author's id, remark
%    \begin{macrocode}
\newcommand{\deleted}[1][\@empty]{%
	\renewcommand{\Changes@temp}{#1}%
	\Changes@deleted%
}
\newcommand{\Changes@deleted}[2][\@empty]{%
	\Changes@output
		{\Changes@Markup@Deleted{#2}}
		{\@bsphack \expandafter \@esphack}
		{\Changes@temp}
		{#1}%
	\stepcounter{Changes@DeleteCount\Changes@temp}%
}
%    \end{macrocode}
% \end{macro}
%
% \begin{macro}{\replaced}
%  The command formats text as replaced text.
%  It's rather complicated for defining a command with two optional parameters.
%
%  Mandatory arguments: new text and old text.
%  Optional arguments: author's id, remark
%    \begin{macrocode}
\newcommand{\replaced}[1][\@empty]{%
	\renewcommand{\Changes@temp}{#1}%
	\Changes@replaced%
}
\newcommand{\Changes@replaced}[3][\@empty]{%
	\Changes@output
		{{\Changes@Markup@Added{#2}}{\Changes@Markup@Deleted{#3}}}
		{#2}
		{\Changes@temp}
		{#1}%
	\stepcounter{Changes@ReplaceCount\Changes@temp}%
}
%    \end{macrocode}
% \end{macro}
%
% \subsection{List of changes}
%
% \begin{macro}{\changes@chopline}
%  Auxiliary command for reading the content of the loc-files.
%  The content is read line by line.
%  One line is parsed with this macro, the order of entries is: id, color, name, added, deleted, replaced.
%  The contents have to be separated by a semicolon.
%    \begin{macrocode}
\def\changes@chopline#1;#2;#3;#4;#5;#6 \\{
	\def\Changes@InID{#1}
	\def\Changes@InColor{#2}
	\def\Changes@InName{#3}
	\def\Changes@InAdded{#4}
	\def\Changes@InDeleted{#5}
	\def\Changes@InReplaced{#6}
}
%    \end{macrocode}
% \end{macro}
%
% \begin{macro}{\listofchanges}
%  This command outputs a list of changes, sorted by authors.
%  The values are read from the loc-file, if it exists.
%  If no loc-file exists, an according message is generated.
%    \begin{macrocode}
\newcommand{\listofchanges}{%
	\ifthenelse{\boolean{Changes@optiondraft}}
		{
			\section*{\listchangesname}
			\IfFileExists{\jobname.\Changes@extension}
				{
					\newboolean{Changes@MoreLines}
					\setboolean{Changes@MoreLines}{true}
					\newread\Changes@InFile
					\openin\Changes@InFile = \jobname.\Changes@extension
					\whiledo{\boolean{Changes@MoreLines}}{
						\read\Changes@InFile to \Changes@Line
						\ifeof\Changes@InFile
							\setboolean{Changes@MoreLines}{false}
						\else
							\expandafter\changes@chopline\Changes@Line\\
							\begin{tabbing}
								mm\=mmmmmm\=\kill
								{
									\ifthenelse{\boolean{Changes@colored}}
										{\color{\Changes@InColor}}
										{}
									\ifthenelse{\equal{\Changes@InID}{\@empty}}
										{\changesauthorname: \changesanonymousname}
										{
											\changesauthorname: \Changes@InID
											\ifthenelse{\equal{\Changes@InName}{\@empty}}
												{}
												{(\Changes@InName)}
										}
								}\\
								\>\changesaddname:\>\Changes@InAdded\\
								\>\changesdeletename:\>\Changes@InDeleted\\
								\>\changesreplacename:\>\Changes@InReplaced\\
							\end{tabbing}
						\fi
					}
					\closein\Changes@InFile
				}{
					\emph{\changesnoloc}
					\PackageWarning{changes}{LaTeX rerun needed for list of changes}
				}
		}{}
}
%    \end{macrocode}
% \end{macro}
%
%  At the end of the document: write the list of changes in the loc-file, therefore open file, write values, close file.
%  Changes are written as \LaTeX-formatted text, so they can simply be read via \chcommand{input}.
%
%  The order of entries is: id, color, name, added, deleted, replaced.
%  The contents have to be separated by a semicolon.
%    \begin{macrocode}
\AtEndDocument{
	\newwrite\Changes@OutFile
	\immediate\openout\Changes@OutFile = \jobname.\Changes@extension
	\setcounter{Changes@iAuthor}{0}
	\whiledo{\value{Changes@iAuthor} < \value{Changes@AuthorCount}}{
		\stepcounter{Changes@iAuthor}
		\def\Changes@ID{\@nameuse{Changes@AuthorID\theChanges@iAuthor}}
		\immediate\write\Changes@OutFile{\Changes@ID;%
			\@nameuse{Changes@AuthorColor\Changes@ID};%
			\@nameuse{Changes@AuthorName\Changes@ID};%
			\the\value{Changes@AddCount\Changes@ID};%
			\the\value{Changes@DeleteCount\Changes@ID};%
			\the\value{Changes@ReplaceCount\Changes@ID}}
	}
	\closeout\Changes@OutFile
}
%    \end{macrocode}
%
%    \begin{macrocode}
%</changes>
%    \end{macrocode}
%
% \PrintChanges
% \PrintIndex
%
%\Finale
\endinput

